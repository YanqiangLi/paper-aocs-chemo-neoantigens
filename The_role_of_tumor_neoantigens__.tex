There is increasing appreciation for the importance of tumor neoantigens in anti-tumor T-cell responses\cite{Schumacher_2015}. Patients with more neoantigens have better overall survival\cite{Brown_2014} and are more likely to respond to checkpoint blockade immunotherapy\cite{Van_Allen_2015,Rizvi_2015}. Several chemotherapies including platinum compounds have long been known to be mutagenic\cite{Hannan_1989} and recent studies have found evidence for chemotherapy-induced mutations in human cancers\cite{Murugaesu_2015,Johnson_2013}. While there has been much study on how chemotherapy may enhance\cite{Hato_2012} or weaken\cite{Litterman_2013} an anti-tumor immune response, the extent to which mutagenic chemotherapies induce tumor neoantigens has not been well assessed.

The Australian Ovarian Cancer Study performed whole genome sequencing on 115 cancer samples from 93 donors with high grade serous ovarian carcinomas\cite{Patch_2015}. Eighty samples were taken before chemotherapy and 35 after combination carboplatin/paclitaxel and other chemotherapy treatments. The primary analysis found that the recurrence samples harbored more somatic mutations than the primary samples and exhibited evidence of chemotherapy-associated mutations. Here we extend these results by quantifying the change in mutational and neoantigen burden with treatment and estimating the number of neoantigens attributable to chemotherapy-associated mutational signatures.

Our analysis is made possible by two recent reports of mutations detected in chemotherapy-exposed \textit{C. Elegans} organisms\cite{Meier_2014} and a \textit{G. Gallus} (chicken) cell line \cite{Szikriszt_2016}.

in widtype and various knockout contexts, and three signatures from a chicken cell line exposed to cisplatin, cyclophosphamide, or etoposide


TCGA neoantigen predictions: \cite{Brown_2014} (median of 3 neoantigens per tumor) \cite{Rooney_2015} (only 7\% of ovarian tumors have >20 mutations predicted to generate neoantigens)


We set out to answer two questions: how many more mutations and neoantigens are there post-treatment, and what are the likely mutagenic mechanisms that are causing them

Several aspects of this dataset require careful analysis. First, as adjuvant paclitaxel/carboplatin therapy is standard of care, there are no relapse samples that were not exposed to chemotherapy. Therefore, the effects of chemotherapy cannot be statistically separated from those of surgery, drift, and relapse, motivating the mutational signature approach we use here. Second, 76/80 pre-treatment samples are solid tumor, whereas 24/35 post-treatment samples are drained ascitic fluid. This imbalance occurs because surgery is rarely performed for relapsed disease. We find evidence that ascites samples tend to harbor more detectable mutations than solid tumors, a phenomenon Patch et al suggest may be due to increased mixing of subclones in ascites. This must be accounted for if one is to compare the mostly solid tumor pre-treatment samples to predominantly ascitic post-treatment samples. We propose a Bayesian regression model to control for tissue type and other variables. Third, 12/93 donors in the cohort have both pre-treatment and post-treatment samples; the remaining donors have only treated or only untreated samples. In assessing changes in mutations after treatment, we would like to maximize statistical power with a paired (within-subject) analysis when possible while also making use of the unpaired samples. The Bayesian model we propose therefore includes a hierarchical component that models donor-specific effects.
