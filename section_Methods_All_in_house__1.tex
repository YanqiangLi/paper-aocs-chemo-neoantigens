\section*{Methods}
All in-house developed tools are available at https://github.com/hammerlab. Data and analysis notebooks are available at https://github.com/hammerlab/ovarian-paired-analyses.

\subsection*{Next-generation sequencing}
We used the original mutation calls generated by \cite{Patch_2015}. DNA and RNA sequencing reads were downloaded from the European Genome-phenome Archive (accession EGAD0000100087). We considered a mutation to be present in a sample if it was called in any sample from the same donor and if there were more than 10 DNA reads supporting the alternate allele.

\subsection*{Bayesian modeling}
If one makes the assumption that adjuvant and neoadjuvant treatments have identical effects on mutational burden, then in principle the five neoadjuvant-treated primary samples enable a disentangling of the effects of treatment and relapse time-point, but such a model had large uncertainties and proved inconclusive (Figure~\ref{fig:bayesian_model_effects_separate}). 

This model integrated information from paired pre- and post-treatment samples and unpaired samples, controlling for sample tissue type (solid or ascites), sample purity, the number of samples from each donor, and allowing for an interaction (effect modification) between tissue type and 
 
\subsection*{HLA typing and MHC binding prediction}
HLA typing was performed using a consensus of seq2HLA\cite{Boegel_2012} and OptiType\cite{Szolek_2014} across the available samples for each donor. Variant annotation was performed using Varcode (https://github.com/hammerlab/varcode). Class I MHC binding predictions were done with NetMHCpan 2.8\cite{Lundegaard_2008}.

\subsection*{Mutational signatures}
Most previous studies have done de-novo signature extraction then compared to existing signatures. We instead deconvolve directly onto existing signatures.

Both ~\cite{Meier_2014} and ~\cite{Szikriszt_2016} sequenced replicates of chemotherapy-treated and untreated (control) organisms. To identify a mutational signature associated with treatment, we must split the mutations observed in the treated group into background (i.e. those we would expect to see without treatment) and treatment effects. We do this for each study and chemotherapy drug separately.

Let $C_{i,j}$ be the number of mutations observed in experiment $0 \leq i \lt E$ for mutational trinucletoide context $0 \leq j \lt 96$. Let $s_i \in \{0,1\}$ be 1 if the treatment was administered in experiment $i$ and 0 if it was a control. We estimate the number of mutations in each context typically arising due to background (non-treatment) processes $B_j$ and the number due to treatment $T_j$ according to the model:

\[
C_{i,j} \sim \mathit{Poisson}(B_j + s_i T_j)
\]

We fit this model using Stan\cite{Gelman_2015} with a uniform (improper) prior on the entries of $B$ and $T$. The mutational signature of treatment $N$ was calculated from a point estimate of $T$ as:

\[
N_j = \left ( \frac{T_j}{\sum_{j'}{T_{j'}}} \right ) \left ( \frac{h_j}{a_j} \right )
\]

where $h_j$ and $m_j$ are the number of times the reference trinucleotide $j$ occurs in the human and animal (\textit{C. Elegans} or \textit{G. Gallus}) genomes, respectively.

The signature deconvolution itself was performed with the deconstructSigs\cite{Rosenthal_2016} package using the following parameters passed to \texttt{whichSignatures()}: \texttt{contexts.needed=TRUE}, \texttt{signature.cutoff=0.05}, \texttt{tri.counts.method = "default"}.

\subsection*{Counting neoantigens attributable to mutational signatures}
To estimate the number of SNVs generated by a signature, we summed the posterior probability that the signature generated each mutation over the mutations in the sample. For neoantigens, we weighted the sum by the number of neoantigens resulting from each SNV.

Let $H_{s,j}$ be the input signatures, giving weights for each signature $s$ on mutational context $j$. Let $D_{i,s}$ be the signature deconvolution result, giving the contribution of signature $s$ to sample $i$. Suppose a mutation occurs in context $c_m$ and sample $l_m$. Let $\Pr[s \mid c_m]$ be the probability that signature $s$ gave rise to this mutation. We calculate this using Bayes' Rule:

\[

\Pr[s \mid c_m] = \frac{\Pr[c_m \mid s] \Pr[s]}{\sum_{s'}{\Pr[c \mid s']\Pr[s']}} = \frac{H_{s,c_m} D_{l_m,s}}{\sum_{s'}{H_{s',c_m} D_{l_m,s'}}}

\]
