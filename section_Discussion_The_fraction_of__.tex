\section*{Discussion}

The fraction of cancer cells harboring a neoantigen may be critical in its ability to be targeted by a T cell response \cite{McGranahan_2016}.

 This suggests the increased burden at relapse in these Ovarian cancer samples is not predominantly due to direct mutagenic impact of the adjuvant chemotherapy, in contrast to a recent report for neo-adjuvant treated Esophageal cancer.
 
 The signature contributions to the primary tumors were not significantly different between high-VAF and low-VAF mutations, suggesting that the mutagenic processes at work in the primary samples were not undergoing change at the time of primary surgery. The decrease in mutations attributable to Signatures 1 and 3 in the unique-to-relapse mutations is therefore potentially an indication of a therapy-driven shift. However, 

We note that the C(C>T)C mutations found at higher rates in the relapse samples do not correspond to the signature found in \textit{C. Elegans} (supplementary figure).