The Australian Ovarian Cancer Study (AOCS)\cite{Patch_2015} performed whole genome sequencing on 92 high grade serous Ovarian cancer tumors, including 15 donors with matched primary and acquired resistance samples. Their analysis found that the recurrence samples harbored more somatic mutations than the primary samples and were enriched for mutations in a $C(C \gt T)C$ context, suggestive of a possible impact of chemotherapy. We extend this analysis by quantifying the number of potential neoantigen-generating mutations in the primary and recurrence samples and testing whether mutational signature deconvolution attributes the unique-to-recurrence mutations to the \textit{C. Elegans} cisplatin signature. We include three additional donors from TCGA and one donor from our institution in our analysis (\ref{tab:cohort}).

TCGA neoantigen predictions: \cite{Brown_2014} (median of 3 neoantigens per tumor) \cite{Rooney_2015} (only 7\% of ovarian tumors have >20 mutations predicted to generate neoantigens)

Ideally we would have some non-chemo treated surigical patients to serve as control. As platinum chemo is standard of care, we have to use signatures to figure out if it's likely due to chemo.

As surgery is not usually performed for relapsed disease, the AOCS relapsed samples consist predominantly of drained ascites samples. These are a potential confounder and needs to be handled carefully.

We set out to answer two questions: how many more mutations and neoantigens are there post-treatment, and what are the likely mutagenic mechanisms that are causing them