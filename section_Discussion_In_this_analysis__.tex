\section*{Discussion}
In this analysis of neoantigens predicted from DNA and RNA sequencing of ovarian cancer tumors and ascites samples, relapse samples obtained after chemotherapy exposure had a median of 90\% more expressed neoantigens than untreated primary samples. However, our proposed chemotherapy mutational signatures accounted for no more than 16\% of the expressed neoantigen burden in any sample. Most of the increase was instead attributable to mutagenic processes already at work in the primary samples, including COSMIC \textit{Signature (3) BRCA} and \textit{Signature (8) Unknown etiology}. Our results are in contrast to a study of NACT temozlomide-treated glioma, in which it was reported that over 98\% of mutations detectable with bulk sequencing in some samples were attributable to temozolomide ~\cite{Johnson_2013}. Whether this difference is due to the drug used or disease biology requires further study.

% This data suggests that mutagenesis from chemotherapy is likely not a dominant factor in this effect. 

Detection of the cyclophosphamide and cisplatin signatures from the \textit{G. Gallus} experiments showed some correlation with clinical treatment, whereas the \textit{G. Gallus} etoposide and \textit{C. Elegans} cisplatin signatures were not detected in chemotherapy-exposed samples. Many treated samples showed no chemotherapy signatures; when chemotherapy signatures were detected, they were found at levels close to the 6\% detection threshold. In the case of cyclophosphamide, the deconvolution of all mutations from all samples identified the signature in 4/10 samples treated with cyclophosphamide and 4/104 unexposed samples. However, when we focused on mutations detected uniquely in the post-treatment paired samples, 6/8 samples exposed only to non-cyclophosphamide chemotherapies exhibited the signature. As it was rarely detected in pre-treatment samples, we suggest that the cyclophosphamide signature present in these post-treatment samples may reflect the effect of other chemotherapy, such as carboplatin, paclitaxel, doxorubicin, or gemcitabine. Analysis of the paired pre- and post-treatment samples indicated that the \textit{G. Gallus} cisplatin signature was specific for cisplatin rather than carboplatin exposure, suggesting that carboplatin may induce fewer mutations or mutations with a different signature than cisplatin. The \textit{C. Elegans} cisplatin signature may be less accurate than the \textit{G. Gallus} cisplatin signature because it was derived from fewer mutations (784 vs. 2633) and from experiments of \textit{C. Elegans} in various knockout backgrounds, which may not be relevant to these clinical samples. While only SNVs are accounted for by mutational signatures, an increase in indels and cisplatin-associated dinucleotide variants was observed in relapse/treated samples, but these variants remained relatively rare and generated less than 13\% of the predicted neoantigen burden in every sample. Etoposide-induced mutations may be difficult to detect because in the \textit{G. Gallus} experiments they occurred at a more uniform distribution of mutational contexts and at a much lower overall rate than mutations induced by cisplatin or cyclophosphamide. Importantly, only one patient in this cohort received etoposide.

% Etoposide-induced mutations may be difficult to detect because in the \textit{G. Gallus} experiments they occurred at a more uniform distribution of mutational contexts and at a much lower overall rate than mutations induced by cisplatin or cyclophosphamide.

The observed association between mutational signatures and clinical exposures gives some confidence that our analysis captures the effect of chemotherapy, but, as the preclinical signatures may differ from actual effects in patients, chemotherapy-induced mutations could be erroneously attributed to non-chemotherapy signatures. This would result in an underestimation of the impact of chemotherapy. We note, however, that the signatures dominant in the primary/untreated samples, COSMIC Signatures (1), (3), and (8), also account for most of the SNVs in relapse/treated samples. It appears unlikely that chemotherapy-induced mutations would happen to match the same signatures operative prior to treatment.

%Relapse/treated samples had a median of 26\% (95\% CI 23--27) of their SNVs attributed to other signatures, compared to 29\% (26--30) for primary/untreated samples.

% If this were a significant effect, however, we would expect to see more mutations attributed to signatures other than chemotherapy and those active in the primaries (COSMIC Signatures 1, 3, and 8) in the relapse/treated samples. We do not see such an increase (see ``Other SNV'' in Figure~\ref{fig:sources}). 

NACT-treated tumors, which were exposed to chemotherapy as large tumors and for a short duration (typically 3 cycles), did not show increased mutation or neoantigen burden over untreated samples and had very few mutations attributed to chemotherapy. This is likely because neoadjuvant chemotherapy-induced mutations remain at undetectable allelic fractions without the population bottleneck created by surgery and/or the multiple lines of chemotherapy provided in the adjuvant setting.

% The observation that the NACT samples trended toward fewer expressed neoantigens raises the possibility of immunoediting after chemotherapy~\cite{Dunn_2002}, consistent with suggestions that paclitaxel/carboplatin and other chemotherapies may enhance T cell infiltration and cytotoxicity in solid tumors~\cite{Demaria2001,Wu_2009,Pfannenstiel_2010,Hodge_2013}. However, this observation is based on only five samples, none of which were paired with a pre-treatment sample, and our analysis method (hard thresholds on the count of RNA reads at variant positions) is sensitive to changes in expression of unrelated genes. Larger cohorts of matched biopsies and post-NACT tumor samples are likely required to investigate this question.

% Such subclonal neoantigens may be unable to support an effective anti-tumor immune response~\cite{McGranahan_2016}.

% STATEMENT_MEDIAN_NEOANTIGENS

We predicted a median of 64 (50--75) expressed MHC I neoantigens across all samples in the cohort, significantly more than the median of 6 recently reported by Martin et al. in this disease ~\cite{Martin_2016}. However, Martin et al. did not consider indels, MNVs, or multiple neoantigens that can result from the same missense mutation, used a 100nm instead of 500nm MHC I binding threshold, used predominantly lower quality (50bp) sequencing, and only explicitly considered HLA-A alleles. Predicted neoantigen burden is best considered a relative measure of tumor foreignness, not an absolute quantity readily comparable across studies.

This study has several important limitations. As it is based on bulk DNA sequencing of heterogeneous clinical samples, the analysis is limited to neoantigens arising from mutations that are present in at least 5-10\% of the cells in a sample. Data from Patch et al. suggests that even late-stage disease remains polyclonal, therefore potentially obscuring the impact of chemotherapy on the tumor genome. While we may have been unable to detect subclonal mutations due to the depth of whole genome sequencing, it is expected that such clones would be unable to trigger an anti-tumor immune response that is effective against the bulk of the tumor~\cite{McGranahan_2016}. Additionally, while the number of mutations attributed to signatures other than chemotherapy and those active in the primaries (COSMIC Signatures 1, 3, and 8) suggest that the preclinical signatures capture most chemotherapy-induced mutations, this reasoning assumes that chemotherapy does not induce mutations that are erroneously attributed to COSMIC Signatures 1, 3, or 8. Experiments using human cell lines exposed to the range of chemotherapies used in recurrent ovarian cancer may be needed to fully address this question. A further limitation is that this study does not consider neoantigens resulting from structural rearrangements such as gene fusions. Finally, this study relies on only 35 post-chemotherapy samples.