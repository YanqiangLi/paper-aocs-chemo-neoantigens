\section*{Discussion}
In this analysis of neoantigens predicted from DNA and RNA sequencing of ovarian cancer tumors and ascites samples, relapse samples obtained after AMCT had a median of 90\% more expressed neoantigens than untreated primary samples. However, chemotherapy mutational signatures accounted for no more than 16\% of the expressed neoantigen burden in any sample, suggesting that mutagenesis from chemotherapy is not a dominant factor in this effect. Mutagenic processes already at work in the primary samples, including COSMIC signatures \textit{(3) BRCA} and \textit{(8) Unknown etiology}, instead accounted for most of the increase in neoantigen burden after chemotherapy.

Detection of the cyclophosphamide and cisplatin signatures extracted from the \textit{G. Gallus} experiments correlated with clinical treatment, whereas the \textit{G. Gallus} etoposide and \textit{C. Elegans} cisplatin signatures were detected in no chemotherapy-exposed samples. In the deconvolution of all mutations from all samples, the cyclophosphamide signature was detected in 4/10 samples treated with cyclophosphamide and 4/104 unexposed samples. However, when we focused on mutations detected uniquely in the post-treatment paired samples, 6/8 samples exposed only to non-cyclophosphamide chemotherapies exhibited the signature. As it was rarely detected in pre-treatment samples, the cyclophosphamide signature may be capturing part of the effect of another chemotherapy, such as carboplatin, paclitaxel, doxorubicin, or gemcitabine. The \textit{G. Gallus} cisplatin signature was specific for cisplatin exposure, but not for exposure to the related compound carboplatin, suggesting that carboplatin may induce fewer mutations or mutations with a different signature than cisplatin. Etoposide-induced mutations may be difficult to detect because in the \textit{G. Gallus} experiments they occurred at a more uniform distribution of mutational contexts and at a much lower overall rate than mutations induced by cisplatin or cyclophosphamide. The \textit{C. Elegans} cisplatin signature may be less accurate than the \textit{G. Gallus} cisplatin signature because it was derived from fewer mutations (784 vs. 2633) and from experiments of \textit{C. Elegans} in various knockout backgrounds, which may not be relevant to these clinical samples.

The observed association between mutational signatures and clinical exposures gives some confidence that our analysis captures the effect of chemotherapy, but to the extent the preclinical signatures differ from actual effects our analysis would underestimate the impact of chemotherapy. However, in this scenario we would expect post-chemotherapy samples to show an increase in the fraction of mutations attributed to signatures other than chemotherapy and those active in the primaries (COSMIC Signatures 1, 3, and 8), as chemotherapy-induced mutations would likely be erroneously attributed to other signatures. We do not see such an increase (``Other SNVs'' in Figure~\ref{fig:sources}). Samples taken after adjuvant-chemotherapy attributed a median of 26\% (95\% CI 23--27) of SNVs to ``other'' signatures, compared to 29\% (26--30) in treatment naive samples.

NACT-treated tumors, which were exposed to chemotherapy prior to surgery, did not show increased mutation or neoantigen burden over untreated samples and exhibited extremely few mutations attributed to chemotherapy. This is likely because chemotherapy-induced mutations remain at undetectable allelic fractions without the population bottleneck created by surgery. Such subclonal neoantigens may be unable to support an effective anti-tumor immune response~\cite{McGranahan_2016}. The observation that these samples trended toward fewer expressed neoantigens raises the possibility of immunoediting after chemotherapy~\cite{Dunn_2002}, an idea consistent with suggestions that paclitaxel/carboplatin and other chemotherapies may enhance T cell infiltration and cytotoxicity in solid tumors~\cite{Demaria2001,Wu_2009,Pfannenstiel_2010,Hodge_2013}. However, this observation is based on only five samples, none of which were paired with a pre-treatment sample, and our analysis method (hard thresholds on the count of RNA reads at variant positions) is sensitive to changes in expression of unrelated genes. Larger cohorts of matched biopsies and post-NACT tumor samples are likely required to investigate this question.

Our results are in contrast to a study of neoadjuvant temozlomide-treated glioma, in which it was reported that over 98\% of mutations detectable in bulk sequencing in some samples were attributable to temozolomide ~\cite{Johnson_2013}. Whether this difference is due to the drug used or disease biology requires further study.

% STATEMENT_MEDIAN_NEOANTIGENS

We predicted a median of 64 (50--75) expressed MHC I neoantigens across all samples in the cohort, significantly more than the median of 6 reported in a recent study of this disease by Martin et al.~\cite{Martin_2016}. However, Martin et al. did not consider indels, MNVs, or multiple neoantigens that can result from the same missense mutation, used a 100nm instead of 500nm MHC I binding threshold, used predominantly lower quality (50bp) sequencing, and only explicitly considered HLA-A alleles. Predicted neoantigen burden is best considered a relative measure of tumor foreignness, not an absolute quantity readily comparable across studies.

This study has several important limitations. As it is based on bulk DNA sequencing of heterogeneous clinical samples, the analysis is limited to neoantigens arising from mutations present in at least 5-10\% of the cells in a sample. Additionally, while the number of mutations attributed to signatures other than chemotherapy and those active in the primaries (COSMIC Signatures (1), (3), or (8)) suggest that the preclinical signatures capture most chemotherapy-induced mutations, this reasoning assumes that chemotherapy does not induce mutations that are erroneously attributed to COSMIC Signatures 1, 3, or 8. Experiments using human cell lines exposed to chemotherapy may be needed to fully address this question. A further limitation is that this study does not consider neoantigens resulting from structural rearrangements such as gene fusions. Finally, this study relies on only 35 post-chemotherapy samples, a reflection of the difficulty of acquiring samples from patients with multiply recurring ovarian cancer, as well in obtaining primary samples after neoadjuvant treatment.