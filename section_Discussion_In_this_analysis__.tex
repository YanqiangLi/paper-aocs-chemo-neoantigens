\section*{Discussion}
In this analysis of the neoantigens predicted from DNA and RNA sequencing of ovarian cancer samples, we observed that relapse samples obtained after surgery and adjuvant chemotherapy had a median of 90\% more expressed neoantigens than untreated primary samples. However, mutagenesis from chemotherapy appears not to be the dominant factor in this effect, as chemotherapy mutational signatures accounted for no more than 16\% of the expressed neoantigen burden in any sample. Mutagenic processes already at work in the primary samples, including COSMIC \textit{Signature (3)}, associated with BRCA disruption, and \textit{Signature (8)}, of unknown etiology, instead account for most of the increase in neoantigen burden after chemotherapy.

Neoadjuvant-treated tumors, which were exposed to chemotherapy prior to surgery, did not show increased mutation or neoantigen burden over untreated samples and exhibited extremely few mutations attributed to chemotherapy. This is likely because chemotherapy-induced mutations remain at undetectable allelic fractions in the neoadjuvant setting, where there is no surgery-induced population bottleneck. Such subclonal neoantigens may be unable to support an effective anti-tumor immune response~\cite{McGranahan_2016}.

The surprising observation that neoadjuvant-treated samples trended toward fewer expressed neoantigens compared to untreated samples raises the possibility of immunoediting after chemotherapy~\cite{Dunn_2002}. This idea is consistent with studies showing that paclitaxel/carboplatin and other chemotherapies can enhance T cell infiltration and cytotoxicity in solid tumors~\cite{Demaria2001,Wu_2009,Pfannenstiel_2010,Hodge_2013}. However, our observation is based on only five samples and our analysis method (hard thresholds on the count of RNA reads at variant positions) is sensitive to changes in expression of unrelated genes. Larger cohorts of matched pre-chemotherapy biopsies and post-chemotherapy tumor samples would likely be required to definitely investigate this question.

Our results are in contrast to a recent study of neoadjuvant temozlomide-treated glioma~\cite{Johnson_2013}, which reported that over 98\% of mutations detectable in bulk sequencing in some samples were attributable to temozolomide. Whether this difference is due to the drug used or disease biology will require further study.

% STATEMENT_MEDIAN_NEOANTIGENS
We predicted a median of 64 (50--75) expressed MHC I neoantigens across all samples in the cohort, significantly more than the median of 6 reported in a recent study of this disease by Martin et al.~\cite{Martin_2016}. However, we note that Martin et al. did not consider indels, MNVs, or multiple neoantigens that can result from the same missense mutation, used a 100nm instead of 500nm MHC I binding threshold, used predominantly lower quality (50bp) sequencing, and only explicitly considered HLA-A alleles. We suggest that counts of predicted MHC I binding peptides are best considered a relative measure of tumor foreignness, not an absolute quantity readily comparable across studies.

Two of the four putative chemotherapy signatures tested correlated with the relevant chemotherapy exposure in patients. These were the signatures for cyclophosphamide and cisplatin extracted from the Szikriszt et al. study of \textit{G. Gallus} cell lines. The cyclophosphamide signature was detected in 4/10 patients treated with cyclophosphamide and 4 / 104 unexposed samples, a significant, albeit inexact, association. Surprisingly, when we focused on the mutations detected uniquely in the post-treatment samples, this association was lost, as 6/8 samples exposed to chemotherapies other than cyclophosphamide exhibited the signature. As the cyclophosphamide signature was rarely detected in pre-treatment samples, it appears possible the signature is incidentally capturing some of the effect of another chemotherapy, such as carboplatin, paclitaxel, doxorubicin, or gemcitabine. The \textit{G. Gallus} cisplatin signature, in contrast, seems highly specific for cisplatin exposure. It was detected only in the unique-to-treated mutations of the two cisplatin-treated samples. As all patients received the related compound carboplatin, it appears carboplatin either induces fewer mutations or induces mutations with a different mutational signature than cisplatin. The remaining signatures were either uncorrelated with clinical record of treatment (\textit{G. Gallus} etoposide) or detected in no samples (\textit{C. Elegans} cisplatin signature). Etoposide-induced mutations may be difficult to detect because, as reported by Szikriszt et al., this drug induces mutations at an approximately uniform rate across mutational contexts, and at a much lower rate than cisplatin or cyclophosphamide. The \textit{C. Elegans} cisplatin signature may be a less accurate approximation of the mutations induced by cisplatin because it was derived from fewer mutations than the \textit{G. Gallus} signature (784 vs. 2633). It also was derived from \textit{C. Elegans} organisms in wildtype and various gene knockout contexts, which may generalize less effectively to human tumors.

This study has several important limitations. It is possible that the animal signatures for chemotherapy exposure may not generalize to clinical human samples, leading us to underestimate of the impact of chemotherapy. If this occurred, however, we would expect post-chemotherapy samples to show an increase in the fraction of mutations attributed to signatures other than chemotherapy and COSMIC signatures (1), (3), and (8), since chemotherapy-induced mutations would likely be erroneously attributed to other signatures. We do not see such an increase (``Other SNVs'' in Figure~\ref{fig:sources}). Samples taken after adjuvant-chemotherapy attributed a median of 26\% (95\% CI 23--27) of SNVs to ``other'' signatures, compared to 29\% (26--30) in treatment-naive samples. It is still possible that chemotherapy may induce mutations that are erroneously attributed to COSMIC Signatures (1), (3), or (8), however, and future experiments using human cell lines exposed to chemotherapy may be needed to fully address this question. A further limitation is that this study does not consider neoantigens resulting from structural rearrangements such as gene fusions. Finally, this study relies on only 35 post-chemotherapy samples, a reflection of the difficulty of acquiring samples from patients with multiply recurring ovarian cancer, as well in obtaining primary samples after neoadjuvant treatment.

% A surprising finding was that solid tumor samples showed a decrease in the number of expressed neoantigens following chemotherapy, despite an increase in the overall number of neoantigens in the DNA. This is in contrast to samples of ascitic fluid, in which the number of expressed neoantigens tracks the increase in total neoantigens following therapy. This finding raises the possibility of immunoediting~\cite{Dunn_2002} at the level of transcription occurring post-chemotherapy in solid tumors. This idea is consistent with studies showing that paclitaxel/carboplatin and other chemotherapies may enhance T cell infiltration and cytotoxicity in solid tumors~\cite{Demaria2001,Wu_2009,Pfannenstiel_2010,Hodge_2013}, whereas ascites are intrinsically immune-suppressed~\cite{Giuntoli2009,Simpson-Abelson2013,Singel2016}. However, this result is based on only four post-treatment solid tumor samples with RNA sequencing, and the analysis method is sensitive to changes in the expression of unrelated genes. Larger cohorts with paired pre- and post-chemotherapy RNA sequencing of solid tumors are needed to determine if transcriptional down-regulation of neoantigens post-treatment is a recurring phenomenon.

% In any study based on bulk sequencing of tumors, detected mutations necessarily occur at relatively high allele fractions (minimum allele fraction of approximately 5\%).

% The neoantigens generated by indels may be over-estimated, as most frame-shifted transcripts are likely targets for nonsense mediated decay. On the other hand, large structural rearrangements such as gene fusions were not considered, and may be immunogenic. As previously mentioned, the hard threshold on read count from RNA-seq to consider a neoantigen expressed is sensitive to changes in the expression of unrelated genes. Finally, a substantial fraction of mutations are unaccounted by signature deconvolution, and this amount increases in the treated samples. Some of these unresolved mutations may be a result of chemotherapy-induced mutations that do not fit the animal-derived signatures.  result

% Our neoantigen identification pipeline is also made available, which includes support for indels and the identification of neoantigens with RNA support. 

% While there has been much study on how chemotherapy may enhance~\cite{Hato_2012,Machiels2001,Hodge2013} or suppress~\cite{Litterman_2013} an anti-tumor immune response, the extent to which it induces tumor neoantigens has not been well assessed.

%  (enrichment in paired treated samples $p=10^{-4}$)
% It is enriched in the chemotherapy-treated samples but shows little correlation with cyclophosphamide administration (one-sided Fischer's exact test $p=0.6$).

% It instead shows a mild trend toward patients treated with gemcitabine, detected in 7/8 gemcitabine-treated samples and 2/6 non-gemcitabine-treated samples ($p=.09$). We hope the approach demonstrated here, as well as the extracted signatures themselves, may be useful to future studies of chemotherapy-induced mutations.

