\section*{Results}

% The treated samples showed more mutations, neoantigens, and, in the case of ascites samples, more expressed neoantigens (Table~\ref{tab:cohort}).

% STATEMENT_TREATMENT_EFFECT
We identified 17,689 mutated peptides predicted to bind autologous MHC class I with affinity 500nm or tighter ~\cite{Sette1994}, which we refer to as neoantigens. All but 21 (0.12\%) neoantigens were private to a single patient.

Solid tissue samples taken after adjuvant chemotherapy and disease recurrence harbored a median 81\% (bootstrap 95\% CI 40--123) more mutations, 124\% (58--191) more neoantigens, and 90\% (40--142) more expressed neoantigens than chemotherapy-naive primary solid tissue samples (Figure~\ref{fig:counts}). The trend was similar for ascites samples, in which treated samples harbored 31\% (14--49), 59\% (14--124), and 90\% (27--190) more mutations, neoantigens, and expressed neoantigens, respectively, than untreated samples. This trend is also apparent from pairwise comparison of samples from the same donors (Figure~\ref{fig:supp_paired}). 

% Solid tissue samples taken after adjuvant chemotherapy and disease recurrence harbored more mutations, neoantigens, and expressed neoantigens than chemotherapy-naive primary solid tissue samples (two-sided Mann-Whitney $p < 0.004$ for all tests, Figure~\ref{fig:counts}). Similar trends were observed for ascites samples ($p = 0.08, 0.11, 0.04$ for increases in mutation, neoantigen, and expressed neoantigen load, respectively), and in a pairwise comparison of samples from the same donors (Figure~\ref{fig:supp_paired}). 

In contrast, neoadjuvant-treated primary samples, which are expected to better isolate any effect of chemotherapy because disease recurrence is not a factor, did not exhibit an increase in mutations, neoantigens, or expressed neoantigens, and in fact trended toward decreased expressed neoantigen burden. Neoadjuvant-treated samples of solid tissue harbored a median of 16 (9--89) expressed neoantigens compared to the median of 44 (39--60) observed in chemotherapy naive samples, a difference attributable to both fewer neoantigens in the DNA (median of 85 (36--306) vs. 130 (108--150)) and a lower rate of expression (median 19 (14--37) vs. 39 (36--42) percent of neoantigens). This difference did not reach significance (Mann-Whitney $p=0.09$), and will require larger cohorts to definitely assess.

We next performed signature deconvolution on the full set of mutations present in each sample (Figure~\ref{fig:supp_signatures} top and middle). Similar to results originally reported by Patch et al, the most prevalent mutational signatures in this cohort were COSMIC \textit{Signature 3}, associated with BRCA disruption, \textit{Signature 8}, of unknown etiology, and \textit{Signature 1}, associated with a slow mutagenic process active in healthy tissue. These signatures together accounted for a mean of 65\% (95\% CI from bootstrap across samples: 63 - 67) of mutations and 58\% (56 - 59) of neoantigens across samples.

The authors of the deconstructSigs package reported that signatures accounting for at least 6\% of the mutations in a sample had good concordance with validation data, but that signatures below this threshold were often false positives. Presence of the cyclophosphamide signature at this threshold was associated with a clinical record of cyclophosphamide treatment (Fischer's exact test $p = 0.001$), occurring in 4/104 non-cyclophosphamide treated samples and 4/10 samples taken after cyclophosphamide treatment. In contrast, the two cisplatin signatures were found in no samples, and the etoposide signature was found only in four pre-treatment samples. Etoposide-induced mutations may be undetectable by signature deconvolution because, as reported in Szikriszt et al, this drug induces mutations at a nearly uniform rate across mutational contexts. 

% STATEMENT_UNIQUE_TO_TREATED_COUNT

For better sensitivity, we next focused on the 14 samples from 12 patients with paired pre- and post-treatment samples. For each patient, we extracted the mutations that had evidence exclusively in the treated samples, requiring more than 30 reads coverage and zero variant reads in the pre-treatment samples. Of 206,766 SNVs in the post-treatment samples for these patients, 93,986 (45\%) satisfied this filter and were subjected to signature deconvolution (Figure ~\ref{fig:signatures}). Within this subgroup, the \textit{G. gallus} cisplatin signature was associated with cisplatin exposure ($p = 0.01$), as it was identified only in the two samples taken after cisplatin therapy. The \textit{C. Elegans} cisplatin signature was detected in no samples, and the cyclophosphamide signature was detected in 3/6 cyclophosphamide-treated samples but, unexpectedly, also in 6/8 non-cyclophosphamide-treated samples. The COSMIC signatures \textit{3} (BRCA disruption) and \textit{8} (unknown etiology) were detected in 14/14 and 9/14 post-treatment samples, respectively, but \textit{Signature 1} (Age) was completely absent, consistent with its association with a slow mutagenic process operative before oncogenesis.

While we cannot determine whether any particular mutation was chemotherapy-induced, we can estimate the fraction of mutations and neoantigens attributable to each signature in a sample (Figures~\ref{fig:sources} and~\ref{fig:supp_sources}). Sample AOCS-092-3-3, a post-treatment sample from a patient who received eight distinct chemotherapeutic agents, the most of any sample in the cohort, showed the greatest combined contribution from chemotherapy signatures, with 21\% (or approximately 3,243 of 15,491) of SNVs, 15\% (9 of 61) neoantigens, and 16\% (5 of 30) expressed neoantigens attributed to a chemotherapy-associated signature.

Signature deconvolution considers only SNVs, but studies of platinum mutagenesis have also reported increases in the rate of dinucleotide variants and indels. Indeed, we observed more multinucleotide variants (MNVs) overall and specifically the platinum-associated MNVs $CT \rightarrow AC$ and $CA \rightarrow AC$ reported by Meier et al~\cite{Meier_2014} in treated patients in both absolute count and as a fraction of mutational burden ($p < 10^{-6}$ for all tests). Sample AOCS-092-3-3, previously found to have the most chemotherapy-signature SNVs, also had the most platinum-associated dinucleotide variants and the second-most MNVs overall. This sample harbored 59 $CT \rightarrow AC$ or $CA \rightarrow AC$ mutations, compared to a mean of 3.2 (2.2--4.4) across all samples. Treated samples also harbored more indels in terms of absolute count ($p=10^{-4}$). Overall, while MNVs and indels generate more neoantigens per mutation than SNVs, they are rare, comprising less than 3\% of the mutational burden and 13\% of the neantigens in every sample (Figure~\ref{fig:sources}), making it unlikely that chemotherapy-induced MNVs and indels have a large impact on neoantigen burden.

% Treated samples harbored a mean of 153 (bootstrap 95\% CI 130--180) MNVs, compared to 62 (56--59) in pre-treatment samples, and specifically showed an increase in the $CT \rightarrow AC$ and $CA \rightarrow AC$ reported by Meier et al~\cite{Meier_2014}


% , the fraction of total mutations from indels  this was not observed as a fraction of burden ($p=0.9$)

It was recently reported that high variant-allele frequency (i.e. clonal) neoantigens are required to elicit a cytotoxic T cell response~\cite{McGranahan_2016}. In our study, in the treated samples, chemotherapy-associated SNVs had on average 11\% (8--15) lower variant allele frequency than other SNVs and were approximately half as likely to be found in the top decile of variant allele frequency.

%Several chemotherapy signatures were detected at a 5\% detection threshold in post-treatment samples, but some were also detected in pre-treatment samples, indicating a substantial false-positive rate. The \textit{C. Elegans} cisplatin \textit{xpf-1} and \textit{fcd-2} knockout signatures were detected in 2/35 and 1/35 post-treatment samples, respectively, and no pre-treatment samples, providing relatively weak evidence that they may be associated with treatment (two-sided Fischer's exact test $p=0.03$). The cyclophosphamide signature was found in 4/80 pre-treatment and 9/35 post-treatment samples ($p=0.003$). However, 5/9 of the post-treatment samples with detection had no clinical record of cyclophosphamide use. The cisplatin \textit{polq-1} knockout signature was similarly found in 3/80 pre-treatment and 6/35 post-treatment samples ($p=0.02$). Other chemotherapy signatures were either not detected in any samples (cisplatin \textit{G. Gallus} signature, other cisplatin \textit{C. Elegans} signatures) or had a high detection rate in the pre-treatment samples (etoposide signature, found in 9/80 pre-treatment samples). The only COSMIC signature with a known chemotherapy association, \textit{Signature 11}, associated with alkylating agents, was not detected in any samples.

% STATEMENT_UNIQUE_TO_TREATED_COUNT
%For better sensitivity, we next focused on the 14 samples from 12 patients with paired pre- and post-treatment samples. For each patient, we extracted the mutations that had evidence exclusively in the treated samples, requiring at least 30 reads coverage and zero variant reads in the pre-treatment samples. Of 222,709 SNV mutations in the post-treatment samples for these patients, 102,377 (46\%) matched this filter and were subjected to signature deconvolution (Figure ~\ref{fig:signatures}). The \textit{G. gallus} cisplatin signature, previously undetected in all samples, was found in the two cisplatin-treated samples present in this analysis (DB: meaning unclear), suggesting it may be a marker of cisplatin, but not carboplatin, exposure (DB: meaning unclear). The \textit{C. Elegans} \textit{fcd-2} knockout was detected in four samples, all having received carboplatin but not cisplatin. COSMIC signatures \textit{3} (BRCA disruption) and \textit{8} (unknown etiology) were detected in 14/14 and 9/14 post-treatment samples, respectively, but \textit{Signature 1} (Age) was completely absent, consistent with its association with a slow mutagenic process operative before oncogenesis.


% We developed a Bayesian model to estimate the magnitude of this increase using paired and unpaired samples and controlling for sample type and purity. In this model, treated samples had an adjusted 57\% more mutations than untreated samples with 95\% confidence that the increase is at least 5\% (Figure~\ref{fig:bayesian_model_effects}).

%found with 95\% certainty that mutational burden increases by at least 5\% with treatment; mean effect was a 57\% increase.

% 95\% posterior probability 95\% posterior probability, this model confirmed that the number of mutations increases by at least 5\% with treatment, with a point estimate of 57\%.

% that the post-treatment timepoint is associated with at least a 5\% increase in mutations, and the mean effect size was 57\%.

% mutations increasing by at least 5\% with treatment; the  the estimated the effect of treatment to be a 57\% increase in mutations (Figure~\ref{fig:bayesian_model_effects}) and with there was a 95\% probability that the increase with treatment was as least 5\%.

% (95\% credible region 0--131)

% STATEMENT_NEOANTIGENS1
% Using sequence-based HLA typing and computational pMHC binding prediction, we identified 18,336 mutated peptides predicted to bind autologous MHC class I with affinity 500nm or tighter ~\cite{Sette1994}, which we refer to as neoantigens. For samples with RNA sequencing, we also counted ``expressed neoantigens,'' defined as neoantigens resulting from mutations with three or more RNA reads supporting the alternate allele. All but 26 (0.14\%) neoantigens were private to a single patient.

%Neoantigens tracked the increase in mutational burden after chemotherapy. In the Bayesian analysis, treated samples had 65\% (95\% credible interval -14--174) more neoantigens, and, for ascites samples, 117\% (3--286) more expressed neoantigens. Interestingly, solid tumor samples showed an increase in neoantigens but a 43\% (2--71) decrease in expressed neoantigens after treatment.
