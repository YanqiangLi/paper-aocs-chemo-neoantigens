\section*{Results}

% The treated samples showed more mutations, neoantigens, and, in the case of ascites samples, more expressed neoantigens (Table~\ref{tab:cohort}).

% STATEMENT_TREATMENT_EFFECT 
For 11/12 donors with paired pre- and post-treatment samples, somatic mutation burden increased after treatment (Figure~\ref{fig:supp_paired}). We developed a Bayesian model to estimate the magnitude of this increase using paired and unpaired samples and controlling for sample type and purity. In this model, treated samples had an adjusted 57\% more mutations than untreated samples with 95\% confidence that the increase is at least 5\% (Figure~\ref{fig:bayesian_model_effects}).

%found with 95\% certainty that mutational burden increases by at least 5\% with treatment; mean effect was a 57\% increase.

% 95\% posterior probability 95\% posterior probability, this model confirmed that the number of mutations increases by at least 5\% with treatment, with a point estimate of 57\%.

% that the post-treatment timepoint is associated with at least a 5\% increase in mutations, and the mean effect size was 57\%.

% mutations increasing by at least 5\% with treatment; the  the estimated the effect of treatment to be a 57\% increase in mutations (Figure~\ref{fig:bayesian_model_effects}) and with there was a 95\% probability that the increase with treatment was as least 5\%.

% (95\% credible region 0--131)

% STATEMENT_NEOANTIGENS1
Using sequence-based HLA typing and computational pMHC binding prediction, we identified 18,336 mutated peptides predicted to bind autologous MHC class I with affinity $\leq 500$nm~\cite{Sette1994}, which we refer to as neoantigens. All but 26 (0.14\%) neoantigens were private to a single donor. Neoantigens tracked the increase in mutational burden after chemotherapy. In the Bayesian analysis, treated samples had 65\% (-14--174) more neoantigens, and, for ascites samples, 117\% (3--286) more expressed neoantigens. Interestingly, solid tumor samples showed an increase in neoantigens but a 43\% (2--71) decrease in expressed neoantigens.

Signature deconvolution was performed individually on all samples onto the 30 mutational signatures curated by COSMIC\cite{364242}, nine signatures extracted from a study of cisplatin-exposed \textit{C. Elegans}~\cite{Meier_2014} in widtype and various knockout contexts, and three signatures from a chicken cell line exposed to cisplatin, cyclophosphamide, or etoposide~\cite{Szikriszt_2016} (Figure~\ref{fig:supp_signatures}). The top signatures were COSMIC \textit{Signature 3}, associated with BRCA disruption, \textit{Signature 8}, of unknown etiology, and \textit{Signature 1}, associated with a slow mutagenic process also active in healthy tissue. These signatures together accounted for over half of the mutations and neoantigens in both treated and untreated samples.

Several chemotherapy signatures were detected at a 5\% detection threshold in post-treatment samples, but most of these were also detected in some pre-treatment samples, indicating a substantial false-positive rate. The \textit{C. Elegans} cisplatin \textit{xpf-1} and \textit{fcd-2} knockout signatures were detected in 2/35 and 1/35 post-treatment samples, respectively, and no pre-treatment samples, suggesting they may be specific but not sensitive markers of treatment (two-sided Fischer's exact test $p=.03$). The cyclophosphamide signature was found in 4/80 pre-treatment and 9/35 post-treatment samples ($p=.003$). However, 5/9 of the post-treatment samples with detection had no clinical record of cyclophosphamide use. The cisplatin \textit{polq-1} knockout signature was similarly found in 3/80 pre-treatment and 6/35 post-treatment samples ($p=.02$). Other chemotherapy signatures were either not detected in any samples (cisplatin \textit{G. Gallus} signature, other cisplatin \textit{C. Elegans} signatures) or had a high detection rate in the pre-treatment samples (etoposide signature, found in 9/80 pre-treatment samples).

% STATEMENT_UNIQUE_TO_TREATED_COUNT
For better sensitivity, we next focused on the 14 samples from 12 donors with paired pre- and post-treatment samples. For each donor, we extracted the mutations that had evidence exclusively in the treated samples, requiring at least 30 reads coverage and zero variant reads in the pre-treatment samples. Of 222,709 SNV mutations in the post-treatment samples for these donors, 102,377 (46\%) matched this filter and were subjected to signature deconvolution (Figure ~\ref{fig:signatures}). The \textit{G. gallus} cisplatin signature, previously undetected in all samples, was found in exactly the two cisplatin-treated samples present in this analysis, suggesting it may be a marker of cisplatin, but not carboplatin, exposure. The \textit{C. Elegans} \textit{fcd-2} knockout was detected in four samples, all having received carboplatin but not cisplatin. COSMIC signatures \textit{3} and \textit{8} were detected in 14/14 and 9/14 post-treatment samples, respectively, but \textit{Signature 1} was completely absent, consistent with its association with a slow mutagenic process operative before oncogenesis.

In summary, the signatures from cisplatin treated \textit{G. Gallus} and \textit{xpf-1} and \textit{fcd-2} knockout \textit{C. Elegans} organisms showed the best concordance with clinical platinum administration. At least one of these signatures were found in 0/80 pre-treatment samples, 3/35 post-treatment samples, and 6/14 paired post-treatment samples considering only mutations undetectable before treatment (enrichment in paired treated samples $p=0.0001$). The \textit{G. Gallus} signature may be specific for cisplatin, whereas the other two signatures are detected in patients treated only with carboplatin. The cyclophosphamide signature is less clear. It is enriched in the chemotherapy-treated samples but shows little correlation with cyclophosphamide administration (one-sided Fischer's exact test $p=0.6$). It instead shows a mild trend toward patients treated with gemcitabine, detected in 7/8 gemcitabine-treated samples and 2/6 non-gemcitabine-treated samples ($p=.09$).

We estimated the number of mutations and neoantigens attributable to each signature (Figures~\ref{fig:sources} and~\ref{fig:supp_sources}). Sample AOCS-092-13, a post-treatment sample from a donor who received eight distinct chemotherapeutic agents, the most of any sample in the cohort, showed the greatest contribution from chemotherapy SNV signatures, at 22\% of mutations and 15\% of neoantigens. The other post-treatment samples showed less than 16\% of mutations and 10\% of neoantigens attributed to chemotherapy-associated SNVs.

Signature deconvolution considers only SNVs, but studies of cisplatin mutagenesis have also reported increases in the rate of dinucleotide variants and indels. Indeed, we observed more MNVs overall and specifically the platinum-associated MNVs $CT \rightarrow AC$ and $CA \rightarrow AC$ reported in \cite{Meier_2014} in treated donors in terms of both absolute counts and as a fraction of overall mutational burden (Mann-Whitney $p < 10^{-6}$ for all tests). Sample AOCS-092-13, previously found to have the most chemotherapy-signature SNVs, also had the most platinum-associated dinucleotide variants and the third-most MNVs overall. This sample harbored 34 $CT \rightarrow AC$ or $CA \rightarrow AC$ mutations, compared to a mean of 5 (4--7) across all samples. While treated samples also harbored more indels in terms of absolute count ($p=.01$), this was not observed as a fraction of burden ($p=0.9$). Overall, while MNVs and indels generate more neoantigens per mutation than SNVs, they are rare, comprising less than 2.5\% mutational burden and contributing less than 10\% of the neantigens in all samples in this cohort (Figure~\ref{fig:sources}), making it unlikely that chemotherapy-induced MNVs and indels have a large impact on post-treatment neoantigen burden.

It was recently reported that high variant-allele frequency (i.e. clonal) neoantigens may be required to elicit a cytotoxic T cell response \cite{McGranahan_2016}. In the treated samples, SNVs likely to be chemotherapy-associated were found at lower variant allele frequencies than other SNVs ($p < 10^{-10}$), but the effect was relatively small. Chemotherapy-associated SNVs had on average 11\% (6--17) lower variant allele frequency than other SNVs and were approximately half as likely to be found in the top decile of variant allele frequency.