\section*{Results}

% STATEMENT_TREATMENT_EFFECT
We identified 17,689 mutated peptides predicted to bind autologous MHC class I with affinity 500nm or tighter ~\cite{Sette1994}, which we refer to as neoantigens. All but 21 (0.12\%) neoantigens were private to a single patient (shared neoantigens are listed in Additional File 6).

Solid tissue samples taken after AMCT and disease recurrence harbored a median of 81\% (bootstrap 95\% CI 40--123) more mutations, 124\% (58--191) more neoantigens, and 90\% (40--142) more expressed neoantigens than chemotherapy-naive primary solid tissue samples (Figure~\ref{fig:counts}). The trend was similar for ascites samples, among which treated samples harbored 31\% (14--49), 59\% (14--124), and 90\% (27--190) more mutations, neoantigens, and expressed neoantigens, respectively, than untreated samples. This trend was also apparent from pairwise comparison of samples from the same donors (Figure~\ref{fig:supp_paired}). 

In contrast, NACT-treated primary samples did not exhibit an increase in mutations, neoantigens, or expressed neoantigens, and in fact trended toward decreased expressed neoantigen burden. NACT-treated samples of solid tissue harbored a median of 16 (9--89) expressed neoantigens compared to the median of 44 (39--60) observed in chemotherapy-naive samples, a difference attributable to both fewer neoantigens in the DNA (median of 85 (36--306) vs. 130 (108--150)) and a lower rate of expression (median 19 (14--37) vs. 39 (36--42)) percent of neoantigens. This difference did not reach significance (Mann-Whitney $p=0.09$), and will require larger cohorts to assess.

We next performed signature deconvolution on the full set of mutations detected in each sample (Figure~\ref{fig:supp_signatures} top and middle). Similarly to results reported by Patch et al., the most prevalent mutational signatures in this cohort were COSMIC \textit{Signature (3)}, associated with BRCA disruption, \textit{Signature (8)}, of unknown etiology, and \textit{Signature (1)}, associated spontaneous deamination of 5-methylcytosine, a slow process active in healthy tissue that correlates with age. These signatures together accounted for a median of 67\% (95\% CI 66 - 69) of mutations, 58\% (56--61) of neoantigens, and 68\% (67--71) expressed neoantigens across samples. These rates did not significantly differ with chemotherapy treatment.

Detection of the cyclophosphamide signature was associated with clinical cyclophosphamide treatment (Bonferroni-corrected Fischer's exact test $p = 0.004$), occurring in 4/10 samples taken after cyclophosphamide treatment, 2/79 pre-treatment samples, and 2/25 samples exposed to chemotherapies other than cyclophosphamide. In contrast, the two cisplatin signatures were found in no samples, and the etoposide signature was found only in four pre-treatment samples.

% STATEMENT_UNIQUE_TO_TREATED_COUNT

For better sensitivity, we next focused on the 14 AMCT-treated samples from the 12 patients with both pre- and post-treatment samples. For each patient, we extracted the mutations that had evidence exclusively in the treated samples, requiring more than 30 reads coverage and zero variant reads in the pre-treatment samples. Of 206,766 SNVs in the post-treatment samples for these patients, 93,986 (45\%) satisfied this filter and were subjected to signature deconvolution (Figure ~\ref{fig:signatures}). Within this subgroup, the \textit{G. gallus} cisplatin signature was associated with cisplatin exposure ($p = 0.04$), as it was identified only in the two samples taken after cisplatin therapy. The \textit{C. Elegans} cisplatin signature was detected in no samples, and the cyclophosphamide signature was detected in 3/6 cyclophosphamide-treated samples, but, unexpectedly, also in 6/8 non-cyclophosphamide-treated samples. These included the two post-treatment samples in which the signature was detected in the earlier analysis plus four additional samples. The COSMIC signatures \textit{(3) BRCA} and \textit{(8) Unknown etiology} were detected in 14/14 and 9/14 post-treatment samples, respectively, but signature \textit{(1) Age} was absent, consistent with its association with a slow mutagenic process operative before oncogenesis.

While we cannot determine whether any particular mutation was chemotherapy-induced, we can estimate the fraction of mutations and neoantigens attributable to each signature in a sample (Figure~\ref{fig:sources}). Sample AOCS-092-3-3, a post-treatment sample from a patient who received eight distinct chemotherapeutic agents, the most of any sample in the cohort, showed the greatest combined contribution from chemotherapy signatures, with 21\% (or approximately 3,200 of 15,491) of SNVs, 15\% (9 of 61) neoantigens, and 16\% (5 of 30) expressed neoantigens attributed to a chemotherapy-associated signature.

Signature deconvolution considers only SNVs, but studies of platinum-induced mutations have also reported increases in the rate of dinucleotide variants and indels. Indeed, we observed more multinucleotide variants (MNVs) overall and specifically the platinum-associated MNVs $CT \rightarrow AC$ and $CA \rightarrow AC$ reported by Meier et al.~\cite{Meier_2014} in treated patients in both absolute count and as a fraction of mutational burden ($p < 10^{-6}$ for all tests). Sample AOCS-092-3-3, previously found to have the most chemotherapy-signature SNVs, also had the most platinum-associated dinucleotide variants and the second-most MNVs overall. This sample harbored 59 $CT \rightarrow AC$ or $CA \rightarrow AC$ mutations, compared to a mean of 3.2 (2.2--4.4) across all samples. Treated samples also harbored more indels in terms of absolute count ($p=10^{-4}$). Overall, while MNVs and indels generate more neoantigens per mutation than SNVs, they are rare, comprising less than 3\% of the mutational burden and 13\% of the neantigens in every sample (Figure~\ref{fig:sources}), making it unlikely that chemotherapy-induced MNVs and indels have a large impact on neoantigen burden.

% It was recently reported that high variant-allele frequency (i.e. clonal) neoantigens are required to elicit a cytotoxic T cell response~\cite{McGranahan_2016}. In our study, in the treated samples, chemotherapy-associated SNVs had on average 11\% (8--15) lower variant allele frequency than other SNVs and were approximately half as likely to be found in the top decile of variant allele frequency.