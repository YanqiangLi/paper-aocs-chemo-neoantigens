
The relapse samples contain on average 27\% (range -17\% to 134\%) more detected somatic mutations than the matched primary samples. The number of potential neoantigens increased accordingly from a mean of XX in the primary samples to XX at relapse. The new mutations undetected in the primary samples tended to be found at lower variant allele frequency (VAF) in the relapse samples than shared mutations (Mann-Whitney $p \lt .01$ in 14/16 samples), indicating the new mutations tended to be more subclonal. However, the size of this effect was relatively small. On average the mutations private to the relapse sample were found at 82\% ( 95\% CI=72-93) of the VAF of shared mutations. This effect was smaller still for mutations predicted to bind MHC I. Within this subset, new mutations were found at 89\% (72-107) of the VAF of shared mutations.

To explore possible etiologies for the increased mutational burden at recurrence, we deconvolved the mutations present in the primaries as well as the unique-to-recurrence mutations into known signatures. We used the 30 signatures curated by COSMIC (http://cancer.sanger.ac.uk/cosmic/signatures) plus an additional signature manually extracted from a study of cisplatin-exposed \textit{C. Elegans} [Meier et al]. Consistent with results originally reported by Patch et al on the AOCS cohort, the mutations were attributable mostly to Signatures 1 and 3, which are associated with Age and BRCA disruption, respectively. These two signatures together accounted for 82\% (75-88) of the mutations in primary samples and 70\% (64-76) of the mutations unique to the relapses. The remaining mutations in the relapses were attributed to different signatures in various samples. Signature 4, associated with smoking, was the only additional signature consistently given nonzero weight in the relapse samples. In the 9 relapse samples in which it was assigned nonzero weight, Signature 4 accounted for 10\% (8-13) of the unique-to-relapse mutations. Two primary samples also showed evidence for Signature 4. Smoking history of these patients was not known.

This suggests the increased burden at relapse in these Ovarian cancer samples is not predominantly due to direct mutagenic impact of the adjuvant chemotherapy, in contrast to a recent report for neo-adjuvant treated Esophageal cancer.
 
The signature contributions to the primary tumors were not significantly different between high-VAF and low-VAF mutations, suggesting that the mutagenic processes at work in the primary samples were not undergoing change at the time of primary surgery. The decrease in mutations attributable to Signatures 1 and 3 in the unique-to-relapse mutations is therefore potentially an indication of a therapy-driven shift. However, 

We note that the C(C>T)C mutations found at higher rates in the relapse samples do not correspond to the signature found in \textit{C. Elegans} (supplementary figure)


\iffalse

\section*{Introduction}

The Australian Ovarian Cancer Study (AOCS)\cite{Patch_2015} performed whole genome sequencing on 92 high grade serous Ovarian cancer tumors, including 15 donors with matched primary and acquired resistance samples. Here we analyze the neoantigen burden burden of these paired tumors 


\section*{Results}
\subsection*{Somatic mutation and neoantigen burden}

As previously reported for the AOCS patients, there were significantly more detectable somatic mutations in the relapse samples over matched primary samples (increase in 15/16 sample pairs; binomial test $p \lt 0.001$). The change ranged from -17\% to 134\% with a mean of 27\%.

We considered the number of mutations resulting in peptides predicted to bind autologous class I MHC with 500nm or stronger affinity (figure \ref{fig:mhc_binding_mutations}). In accordance with the increase in overall mutation burden, we observed an upward trend in potential neoantigens at recurrence. For 10/14 patients, relapse samples harbored more potential neoantigens than matched primary samples; however this trend did not reach significance (p=0.18). The change ranged from -27\% to 126\% with a mean of 29\%. About half of these mutations had evidence for expression in the RNA-seq. For 9/13 donors the number of expressed potentially antigenic mutations increased at relapse (-53\% -- 407\%, mean 18\%) (p=0.27) (figure \ref{fig:expressed_mhc_binding_mutations}). We found no evidence for selection against expressed potential neoantigens at relapse (data not shown).

TODO: results bayesian model that includes therapy and basic patient info to predict number of mutations

% As expected somatic mutation and neoantigen burden were highly correlated with no substantial deviations
% Therapy vs. burden

\subsection*{Clonality}
Most of our primary tumors are solid and most of the relapse are ascites, making it challenging to directly address changes in clonality during the time of treatment. We instead focus on the relapse samples and compare the allelic fractions of private vs. shared mutations.

TODO: sciclone / pyclone / phylowgs number of clones (or entropy of clones?) comparing pre-vs-post treatment

The new mutations undetected in the primary samples tended to be found at lower variant allele frequency (VAF) in the relapse samples than shared mutations (Mann-Whitney $p \lt .01$ in 14/16 samples), indicating the new mutations tended to be more subclonal. However, the size of this effect was relatively small. On average the mutations private to the relapse sample were found at 82\% (bootstrap 95\% CI=72-93) of the VAF of shared mutations. This effect was smaller still for mutations predicted to bind MHC I. Within this subset, new mutations were found at 89\% (72-107) of the VAF of shared mutations.

\subsection*{Mutation signatures}
To explore possible etiologies for the increased mutational burden at recurrence, we deconvolved the mutations present in the primaries as well as the unique-to-recurrence mutations into known signatures. We used the 30 signatures curated by COSMIC (http://cancer.sanger.ac.uk/cosmic/signatures) plus an additional signature manually extracted from a study of cisplatin-exposed \textit{C. Elegans} [Meier et al]. Consistent with results originally reported by Patch et al on the AOCS cohort, the mutations were attributable mostly to Signatures 1 and 3, which are associated with Age and BRCA disruption, respectively. These two signatures together accounted for 82\% (75-88) of the mutations in primary samples and 70\% (64-76) of the mutations unique to the relapses. The remaining mutations in the relapses were attributed to different signatures in various samples. Signature 4, associated with smoking, was the only additional signature consistently given nonzero weight in the relapse samples. In the 9 relapse samples in which it was assigned nonzero weight, Signature 4 accounted for 10\% (8-13) of the unique-to-relapse mutations. Two primary samples also showed evidence for Signature 4. Smoking history of these patients was not known.


\section*{Discussion}

The fraction of cancer cells harboring a neoantigen may be critical in its ability to be targeted by a T cell response \cite{McGranahan_2016}.

\fi
