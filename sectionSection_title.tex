\section*{Results}
\subsection*{Somatic mutation and neoantigen burden}

We first considered total somatic mutation burden. For 15/16 donors, the median number of detectable somatic mutations increased at relapse, a significant effect (binomial test, p \lt 0.001) \ref{fig:total_mutations}. The change ranged from -17\% to 134\% with a mean of 27\%.

We next considered the number of mutations resulting in peptides predicted to bind autologous class I MHC with 500nm or stronger affinity. As expected from the increase in total somatic mutation burden, we observed an upward trend in potential neoantigens at recurrence. For 10/14 patients, relapse samples harbored more potential neoantigens than matched primary samples, a trend that did not reach significance (p=0.18). The change ranged from -27\% to 126\% with a mean of 29\%. About half of these mutations had RNA-seq evidence for expression. For 9/13 donors the number of expressed potentially antigenic mutations increased at relapse (-53\% -- 407\%, mean 18\%), again a trend that did not reach significance (p=0.27).

% As expected somatic mutation and neoantigen burden were highly correlated with no substantial deviations
% Therapy vs. burden

\subsection*{Mutational signatures}
To explore possible etiologies for the increased mutational burden at recurrence, we deconvolved the mutations present in the primaries as well as the unique-to-recurrence mutations into known signatures. We used the 30 signatures curated by COSMIC (http://cancer.sanger.ac.uk/cosmic/signatures) plus an additional signature extracted from a study of C. Elegans after cisplatin exposure [Meier et al]. Consistent with results originally reported by Patch et al on this data, the mutations present in the primary tumors were attributable mostly to Signatures 1 and 3, which are associated with Age and BRCA disruption, respectively. We deconvolved the mutations unique to the recurrence samples. 

We note that the C(C>T)C mutations found at higher rates in the relapse samples do not correspond to the signature found in \textit{C. Elegans} (supplementary figure).

\subsection*{Allelic fractions}



