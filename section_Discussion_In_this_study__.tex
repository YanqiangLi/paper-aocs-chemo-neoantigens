\section*{Discussion}

In this study of pre- and post-chemotherapy ovarian carcinoma samples, we find that neoantigen burden at relapse is frequently double that of the primary tumor. However, in the post-treatment samples where we have the best sensitivity to detect chemotherapy mutations ---those with matched pre-treatment samples--- experimentally-derived chemotherapy signatures accounted for a median 7\% (range 0--22) of mutations and 5\% (range 0--15) of the neoantigens. We conclude that direct mutagenesis from standard chemotherapy regimes typically contributes a minority of the neoantigen burden at relapse in this disease. These results are in contrast to a recent study of neoadjuvant temozlomide-treated glioma\cite{Johnson_2013}, which reported that over 98\% of mutations in some samples were attributable to temozolomide treatment. Whether this difference is due to the drug used, time of administration (adjuvant/neoadjuvant), or disease type will require further study.

% While there has been much study on how chemotherapy may enhance~\cite{Hato_2012,Machiels2001,Hodge2013} or suppress~\cite{Litterman_2013} an anti-tumor immune response, the extent to which it induces tumor neoantigens has not been well assessed.

%  (enrichment in paired treated samples $p=10^{-4}$)
% It is enriched in the chemotherapy-treated samples but shows little correlation with cyclophosphamide administration (one-sided Fischer's exact test $p=0.6$).

% It instead shows a mild trend toward patients treated with gemcitabine, detected in 7/8 gemcitabine-treated samples and 2/6 non-gemcitabine-treated samples ($p=.09$). We hope the approach demonstrated here, as well as the extracted signatures themselves, may be useful to future studies of chemotherapy-induced mutations.

% STATEMENT_MEDIAN_NEOANTIGENS
We predicted a median of 64 (50--75) expressed MHC I neoantigens across all samples in the cohort, significantly more than the median of 6 reported in another recent study of this cancer type\cite{Martin_2016}. However, we note that Martin et al did not consider indels, MNVs, or multiple neoantigens that can result from the same missense mutation, used a 100nm instead of 500nm MHC I binding threshold, used predominantly lower quality (50bp) sequencing, and only explicitly considered HLA-A alleles. We suggest that counts of predicted MHC I binding peptides are best considered a relative measure of tumor foreignness, not an absolute quantity readily comparable across studies.

The animal-derived mutational signatures most consistent with human clinical treatment were from cisplatin-treated wildtype \textit{G. Gallus} and \textit{C. Elegans} in \textit{xpf-1} and \textit{fcd-2} knockout contexts. One or more of these signatures were found in 0/80 pre-treatment samples, 3/35 post-treatment samples, and 6/14 paired post-treatment samples after filtering to mutations undetectable before treatment, highlighting the increase in sensitivity afforded by paired pre- and post-treatment sequencing. The \textit{G. Gallus} signature may be specific for cisplatin, whereas the other two signatures are detected in patients treated only with carboplatin. The cyclophosphamide signature is less clear. It is enriched in the chemotherapy-treated samples but shows little correlation with cyclophosphamide administration.

A surprising finding was that in samples of post-chemotherapy ascitic fluid, the number of neoantigens with evidence for RNA expression tracks the increase in DNA neoantigens, but decreases in solid tumors, despite the overall increase in neoantigens in the DNA. This raises the possibility of immunoediting~\cite{Dunn_2002} at the level of transcription occurring post-chemotherapy in solid tumors. This idea is consistent with studies showing that paclitaxel/carboplatin and other chemotherapies may enhance T cell infiltration and cytotoxicity in solid tumors~\cite{Demaria2001,Wu_2009,Pfannenstiel_2010,Hodge_2013}, whereas ascites are intrinsically immune-suppressed~\cite{Giuntoli2009,Simpson-Abelson2013,Singel2016}. However, this result is based on only four post-treatment solid tumor samples with RNA sequencing, and the analysis method is sensitive to changes in the expression of unrelated genes. Larger cohorts with paired pre- and post-chemotherapy RNA sequencing of solid tumors are needed to determine if transcriptional down-regulation of neoantigens post-treatment is a recurring phenomenon.

This study has several important limitations. It is possible that the animal signatures for chemotherapy exposure may not generalize to human cells, resulting in an underestimate of the impact of chemotherapy. The mutations that could not be accounted for by deconvolution suggest that the extent to which this may occur would probably not affect the overall conclusion, however. In the pre-treatment samples, on average 15\% (range 2--30) of the SNVs were not accounted for by deconvolution; in the post-treatment samples this rises to 20\% (range 10--37). Even if all unaccounted for SNVs were due to chemotherapy, they would still account for fewer neoantigens than COSMIC Signature 3 (BRCA disruption). It is still possible that chemotherapy may induce mutations that are erroneously attributed to an existing signature, however, and future experiments using human cancer cell lines exposed to chemotherapy will be needed to fully address this question. A further limitation is that this study does not consider neoantigens resulting from structural rearrangements such as gene fusions.

% The neoantigens generated by indels may be over-estimated, as most frame-shifted transcripts are likely targets for nonsense mediated decay. On the other hand, large structural rearrangements such as gene fusions were not considered, and may be immunogenic. As previously mentioned, the hard threshold on read count from RNA-seq to consider a neoantigen expressed is sensitive to changes in the expression of unrelated genes. Finally, a substantial fraction of mutations are unaccounted by signature deconvolution, and this amount increases in the treated samples. Some of these unresolved mutations may be a result of chemotherapy-induced mutations that do not fit the animal-derived signatures.  result

% Our neoantigen identification pipeline is also made available, which includes support for indels and the identification of neoantigens with RNA support. 

