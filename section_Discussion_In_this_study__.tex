\section*{Discussion}
In this study of pre- and post-chemotherapy ovarian cancer samples, we observed that for relapse samples taken after surgery and adjuvant chemotherapy, median predicted expressed neoantigen burden from bulk sequencing is nearly double that of untreated primary tumors. However, chemotherapy mutational signatures accounted for no more than 16\% of the expressed neoantigen burden in any sample, suggesting that mutagenesis from chemotherapy is not the dominant factor in this effect. Mutagenic processes largely already at work in the primary samples, including COSMIC Signatures \textit{(3) BRCA}, and \textit{(8) (Unknown etiology)}, instead account for most of the increase in neoantigen burden after chemotherapy. 

Neoadjuvant-treated tumors, for which chemotherapy treatment is applied in the absense of surgery, did not show increased mutation or neoantigen burden and exhibited extremely few mutations attributed to chemotherapy. This is likely because chemotherapy-induced mutations remain at undetectable allelic fractions in the neoadjuvant setting, where there is no surgery-induced population bottleneck. Whether such subclonal mutations can form the basis for an effective anti-tumor immune response is unclear but there is some evidence they may not\cite{McGranahan_2016}.

Our results are in contrast to a recent study of neoadjuvant temozlomide-treated glioma~\cite{Johnson_2013}, which reported that over 98\% of mutations detectable in bulk sequencing in some samples were attributable to temozolomide. Whether this difference is due to the drug used or disease biology will require further study.

% STATEMENT_MEDIAN_NEOANTIGENS
We predicted a median of 64 (50--75) expressed MHC I neoantigens across all samples in the cohort, significantly more than the median of 6 reported in a recent study of this disease by Martin et al.~\cite{Martin_2016}. However, we note that Martin et al. did not consider indels, MNVs, or multiple neoantigens that can result from the same missense mutation, used a 100nm instead of 500nm MHC I binding threshold, used predominantly lower quality (50bp) sequencing, and only explicitly considered HLA-A alleles. We suggest that counts of predicted MHC I binding peptides are best considered a relative measure of tumor foreignness, not an absolute quantity readily comparable across studies.

Of the four putative chemotherapy signatures tested, two --those of cyclophosphamide and cisplatin extracted from the Szikriszt et al study of \textit{G. Gallus} cell lines-- correlated with the relevant chemotherapy exposure in patients. The cyclophosphamide signature was detected in 4 / 10 patients treated with cyclophosphamide and 4 / 104 unexposed samples, a significant, albeit inexact, association. Surprisingly, when we narrowed our focus to the mutations present only in the post-treatment samples, this association was lost, as 6 / 8 samples exposed to chemotherapies other than cyclophosphamide exhibited the signature. As this signature was very rarely detected in pre-treatment samples, it appears possible the cyclophosphamide signature is incidentally capturing some of the effect of another chemotherapy, such as carboplatin, paclitaxel, doxorubicin, or gemcitabine. The \textit{G. Gallus} cisplatin signature, in contrast, seems highly specific for cisplatin exposure. It was detected only in the unique-to-treated mutations of the two cisplatin-treated samples. As all patients received the similar compound carboplatin, it appears carboplatin either induces much fewer mutations than cisplatin, or induces mutations with a different mutational signature. The remaining signatures, were either uncorrelated with clinical record of treatment (\textit{G. Gallus} etoposide) or detected in no samples (\textit{C. Elegans} cisplatin signature). Etoposide-induced mutations may be undetectable by signature deconvolution because, as reported by Szikriszt et al, this drug induces mutations at a nearly uniform rate across mutational contexts. The \textit{C. Elegans} cisplatin signature may not have been detected because 


Fewer mutations were reported in C. Elegans making its signature likely less accurate

The animal-derived mutational signatures most consistent with human clinical treatment were from cisplatin-treated wildtype \textit{G. Gallus} and \textit{C. Elegans} in \textit{xpf-1} and \textit{fcd-2} knockout contexts. One or more of these signatures were found in 0/80 pre-treatment samples, 3/35 post-treatment samples, and 6/14 paired post-treatment samples after filtering to mutations undetectable before treatment, highlighting the increase in sensitivity afforded by paired pre- and post-treatment sequencing. The \textit{G. Gallus} signature may be specific for cisplatin, whereas the other two signatures are detected in patients treated only with carboplatin. The relevance of the cyclophosphamide signature is less clear. This signature is enriched in the chemotherapy-treated samples but shows little correlation with cyclophosphamide administration.

% A surprising finding was that in samples of post-chemotherapy ascitic fluid, the number of neoantigens with evidence for RNA expression tracks the increase in DNA neoantigens, but decreases in solid tumors, despite the overall increase in neoantigens in the DNA.

A surprising finding was that solid tumor samples showed a decrease in the number of expressed neoantigens following chemotherapy, despite an increase in the overall number of neoantigens in the DNA. This is in contrast to samples of ascitic fluid, in which the number of expressed neoantigens tracks the increase in total neoantigens following therapy. This finding raises the possibility of immunoediting~\cite{Dunn_2002} at the level of transcription occurring post-chemotherapy in solid tumors. This idea is consistent with studies showing that paclitaxel/carboplatin and other chemotherapies may enhance T cell infiltration and cytotoxicity in solid tumors~\cite{Demaria2001,Wu_2009,Pfannenstiel_2010,Hodge_2013}, whereas ascites are intrinsically immune-suppressed~\cite{Giuntoli2009,Simpson-Abelson2013,Singel2016}. However, this result is based on only four post-treatment solid tumor samples with RNA sequencing, and the analysis method is sensitive to changes in the expression of unrelated genes. Larger cohorts with paired pre- and post-chemotherapy RNA sequencing of solid tumors are needed to determine if transcriptional down-regulation of neoantigens post-treatment is a recurring phenomenon.

In any study based on bulk sequencing of tumors, detected mutations necessarily occur at relatively high allele fractions (minimum allele fraction of approximately 5\%).

This study has several important limitations. (TODO: NOte issues of clonal dominance. Might ascites just be subclonal?) It is possible that the animal signatures for chemotherapy exposure may not generalize to human cells, resulting in an underestimate of the impact of chemotherapy. Etoposide treatment, with its relatively flat distribution across mutational contexts, may be indistinguishable from other effects and thus under count. However, etoposide caused way ferwer mutations than cisplatin in G. Gallus, so it is plausable to use cisplatin as an upper bar. The mutations that could not be accounted for by deconvolution suggest that the extent to which this may occur would probably not affect the overall conclusion, however. In the pre-treatment samples, on average 15\% (range 2--30) of the SNVs were not accounted for by deconvolution; in the post-treatment samples this rises to 20\% (range 10--37). Even if all unaccounted for SNVs were due to chemotherapy, they would still account for fewer neoantigens than COSMIC Signature 3 (BRCA disruption). It is still possible that chemotherapy may induce mutations that are erroneously attributed to an existing signature, however, and future experiments using human cancer cell lines exposed to chemotherapy will be needed to fully address this question. A further limitation is that this study does not consider neoantigens resulting from structural rearrangements such as gene fusions. Finally, this study relies on only 35 post-chemotherapy samples, a reflection of the difficulty of acquiring samples from patients with multiply recurring ovarian cancer, as well in obtaining primary samples before neoadjuvant treatment.

% The neoantigens generated by indels may be over-estimated, as most frame-shifted transcripts are likely targets for nonsense mediated decay. On the other hand, large structural rearrangements such as gene fusions were not considered, and may be immunogenic. As previously mentioned, the hard threshold on read count from RNA-seq to consider a neoantigen expressed is sensitive to changes in the expression of unrelated genes. Finally, a substantial fraction of mutations are unaccounted by signature deconvolution, and this amount increases in the treated samples. Some of these unresolved mutations may be a result of chemotherapy-induced mutations that do not fit the animal-derived signatures.  result

% Our neoantigen identification pipeline is also made available, which includes support for indels and the identification of neoantigens with RNA support. 

% While there has been much study on how chemotherapy may enhance~\cite{Hato_2012,Machiels2001,Hodge2013} or suppress~\cite{Litterman_2013} an anti-tumor immune response, the extent to which it induces tumor neoantigens has not been well assessed.

%  (enrichment in paired treated samples $p=10^{-4}$)
% It is enriched in the chemotherapy-treated samples but shows little correlation with cyclophosphamide administration (one-sided Fischer's exact test $p=0.6$).

% It instead shows a mild trend toward patients treated with gemcitabine, detected in 7/8 gemcitabine-treated samples and 2/6 non-gemcitabine-treated samples ($p=.09$). We hope the approach demonstrated here, as well as the extracted signatures themselves, may be useful to future studies of chemotherapy-induced mutations.

