\section*{Discussion}

In this study of pre- and post-chemotherapy ovarian cancer samples, we find that neoantigen burden at relapse is frequently double that of the primary tumor. However, typically less than 5\% of the neoantigens after treatment can be attributed to chemotherapy-induced mutations using experimentally-derived signatures. In the most heavily treated patient, chemotherapy-associated SNVs accounted for 22\% of mutations and 15\% of neoantigens. We conclude that direct mutagenesis from chemotherapy typically contributes a minority of the mutation-derived neoantigen burden at relapse.

The animal-derived mutational signatures most consistent with human clinical treatment were from cisplatin-treated wildtype \textit{G. Gallus} and \textit{C. Elegans} in \textit{xpf-1} and \textit{fcd-2} knockout contexts. One or more of these signatures were found in 0/80 pre-treatment samples, 3/35 post-treatment samples, and 6/14 paired post-treatment samples after filtering to mutations undetectable before treatment, highlighting the increase in sensitivity afforded by paired pre- and post-treatment sequencing. The \textit{G. Gallus} signature may be specific for cisplatin, whereas the other two signatures are detected in patients treated only with carboplatin. The cyclophosphamide signature is less clear. It is enriched in the chemotherapy-treated samples but shows little correlation with cyclophosphamide administration.

%  (enrichment in paired treated samples $p=10^{-4}$)
% It is enriched in the chemotherapy-treated samples but shows little correlation with cyclophosphamide administration (one-sided Fischer's exact test $p=0.6$).

% It instead shows a mild trend toward patients treated with gemcitabine, detected in 7/8 gemcitabine-treated samples and 2/6 non-gemcitabine-treated samples ($p=.09$). We hope the approach demonstrated here, as well as the extracted signatures themselves, may be useful to future studies of chemotherapy-induced mutations.

% STATEMENT_MEDIAN_NEOANTIGENS
We predicted a median of 64 (50--75) expressed MHC I neoantigens across all samples in the cohort, significantly more than the median of 6 reported in another recent study of this cancer type\cite{Martin_2016}. However, we note that Martin et al did not consider indels, MNVs, or multiple neoantigens that can result from the same missense mutation, used a 100nm instead of 500nm MHC I binding threshold, used predominantly lower quality (50bp) sequencing, and only explicitly considered HLA-A alleles. We suggest that counts of predicted MHC I binding peptides are best considered a relative measure of tumor foreignness, not an absolute quantity readily comparable across studies.

A surprising finding was that in samples of post-chemotherapy ascitic fluid, the number of neoantigens with evidence for RNA expression tracks the increase in DNA neoantigens, but decreases in solid tumors, despite the overall increase in neoantigens in the DNA. This raises the possibility of immunoediting~\cite{Dunn_2002} at the level of transcription occurring post-chemotherapy in solid tumors. This idea is consistent with studies showing that paclitaxel/carboplatin and other chemotherapies may enhance T cell infiltration and cytotoxicity in solid tumors~\cite{Demaria2001,Wu_2009,Pfannenstiel_2010,Hodge_2013}, whereas ascites are intrinsically immune-suppressed~\cite{Giuntoli2009,Simpson-Abelson2013,Singel2016}. However, this result is based on only four post-treatment solid tumor samples with RNA sequencing, and the analysis method is sensitive to changes in the expression of unrelated genes. Larger cohorts with paired pre- and post-chemotherapy RNA sequencing of solid tumors are needed to determine if transcriptional down-regulation of neoantigens post-treatment is a recurring phenomenon.

A key limitation of this study is that the mutations induced in \textit{C. Elegans} and \textit{G. Gallus} with chemotherapy exposure may not generalize to human cells, resulting in an under-estimate of the impact of exposure. The mutations that could not be accounted for by deconvolution onto known signatures suggest that this may occur to a modest extent that would not affect the overall conclusion. In the pre-treatment samples, on average 15\% (range 2--30) of the SNVs were not accounted for by deconvolution, whereas in post-treatment samples this rises to 20\% (range 10--37). All else being constant, one may therefore expect that an additional 5\% of mutations (6\% of neoantigens) may be caused by chemotherapy in contexts that the animal signatures do not reflect. Future experiments using human cancer cell lines exposed to chemotherapy will be needed to fully address this question, however. 

% The neoantigens generated by indels may be over-estimated, as most frame-shifted transcripts are likely targets for nonsense mediated decay. On the other hand, large structural rearrangements such as gene fusions were not considered, and may be immunogenic. As previously mentioned, the hard threshold on read count from RNA-seq to consider a neoantigen expressed is sensitive to changes in the expression of unrelated genes. Finally, a substantial fraction of mutations are unaccounted by signature deconvolution, and this amount increases in the treated samples. Some of these unresolved mutations may be a result of chemotherapy-induced mutations that do not fit the animal-derived signatures.  result

% Our neoantigen identification pipeline is also made available, which includes support for indels and the identification of neoantigens with RNA support. 

