\section*{Background}

Several chemotherapies including platinum compounds~\cite{Hannan_1989}, cyclophosphamide~\cite{Anderson_1995}, and etoposide~\cite{NAKANOMYO_1986} exert their effect through DNA damage, and recent studies have found evidence for chemotherapy-induced mutations in post-treatment acute myeloid leukaemia~\cite{Ding_2012}, glioma~\cite{Johnson_2013}, and esophageal adenocarcinoma~\cite{Murugaesu_2015}. With the development of immune checkpoint-mediated therapy\cite{Chen_2013}, there is increasing appreciation for the importance of T cell antigens arising from somatic mutations in anti-tumor immune responses~\cite{Schumacher_2015}. Studies based on bulk-sequencing of tumor samples followed by computational peptide-class I MHC affinity prediction~\cite{Lundegaard_2007} have suggested that tumors with more mutations and predicted mutant MHC I peptide ligands, which we refer to as neoantigens, may be more likely to respond to checkpoint blockade immunotherapy~\cite{Van_Allen_2015,Rizvi_2015}. However, the effect of standard chemotherapy regimes on detectable tumor mutations and resulting neoantigens in ovarian cancer is poorly understood.

% , and it has been suggested that the relatively low neoantigen burden in ovarian cancer may partly explain the limited effectiveness of immunotherapy in this disease~\cite{Martin_2016}

% However, the effect of standard chemotherapy regimes on tumor mutations and resulting neoantigens in both the neoadjuvant and adjuvant settings is unknown.

Investigators associated with the Australian Ovarian Cancer Study (AOCS) performed whole genome and RNA sequencing on 114 cancer samples from 92 patients with high grade serous ovarian carcinoma~\cite{Patch_2015}. Seventy-nine samples were taken before chemotherapy, 5 after neoadjuvant therapy (NACT), and 30 after chemotherapy administered in the adjuvant or metastatic setting (AMCT). Treatment regimes varied but always included platinum-based chemotherapy. In their primary analysis, Patch et al. reported that post-treatment samples harbored more somatic mutations than pre-treatment samples and exhibited evidence of chemotherapy-associated mutations. Here we extend these results by quantifying the mutations and predicted neoantigens attributable to chemotherapy-associated mutational signatures.

The use of mutational signatures is necessary because it is not possible to distinguish chemotherapy-induced mutations from temporal effects when comparing primary and relapse samples by mutation count alone. A mutational signature ascribes a probability to each of the 96 possible single-nucleotide variants, where a variant is defined by its reference base pair, alternate base pair, and base pairs immediately adjacent to the mutation. Signatures have been associated with exposure to particular mutagens, age related DNA changes, and disruption of DNA damage repair pathways due to somatic mutations or germline risk variants in melanoma, breast, lung and other cancers~\cite{Alexandrov2013}, and provide a means of identifying the contribution that chemotherapy may make to the mutations seen in post-treatment samples. For example, the chemotherapy temozolomide has been shown to induce mutations consisting predominantly of $C \rightarrow T$ (equivalently, $G \rightarrow A$) transitions at CpC and CpT dinucleotides~\cite{Johnson_2013}. To perform deconvolution, the single nucleotide variants (SNVs) observed in a sample are tabulated by trinucleotide context, and a combination of signatures, each corresponding to a mutagenic process, is found that best explains the observed counts. Mutational signatures may be discovered \textit{de novo} from large cancer sequencing projects but for smaller studies such as ours it is preferable to deconvolve using known signatures~\cite{Rosenthal_2016}.

In this study, we calculated the number of mutated peptides predicted to bind autologous MHC class I with affinity 500nm or tighter ~\cite{Sette1994}, which we refer to as neoantigens. 

In this study, we assess changes in mutation and neoantigen burden after NACT and AMCT, perform signature deconvolution using putative signatures for chemotherapy exposure derived from preclinical studies, and quantify the contribution of each signature on neoantigen burden. We find that neoantigen burden increases after AMCT and relapse but not after NACT, and that only a small part of the increase is attributable to chemotherapy.

