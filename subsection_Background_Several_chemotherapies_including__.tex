\section*{Background}

Several chemotherapies including platinum compounds~\cite{Hannan_1989}, cyclophosphamide~\cite{Anderson_1995}, and etoposide~\cite{NAKANOMYO_1986} exert their effect through DNA damage, and recent studies have found evidence for chemotherapy-induced mutations in post-treatment acute myeloid leukaemia~\cite{Ding_2012}, glioma~\cite{Johnson_2013}, and esophageal adenocarcinoma~\cite{Murugaesu_2015}. With the development of immune checkpoint-mediated therapy\cite{Chen_2013}, there is increasing appreciation for the importance of T cell neoantigens arising from somatic mutations in anti-tumor immune responses~\cite{Schumacher_2015}. Studies based on bulk-sequencing of tumor samples followed by computational peptide-class I MHC affinity prediction\cite{Lundegaard_2007} have suggested that tumors with more mutations and predicted mutant MHC peptide ligands may be more likely to respond to checkpoint blockade immunotherapy~\cite{Van_Allen_2015,Rizvi_2015}, and it has been suggested that the relatively low neoantigen burden in ovarian cancer may partly explain the limited effectiveness of immunotherapy in this disease~\cite{Martin_2016}. However, the effect of standard chemotherapy regimes on detectable tumor mutations and resulting neoantigens in ovarian cancer is poorly understood.

% Several chemotherapies including platinum compounds~\cite{Hannan_1989}, cyclophosphamide~\cite{Anderson_1995}, and etoposide~\cite{NAKANOMYO_1986} have long been known to be mutagenic, and recent studies have found evidence for chemotherapy-induced mutations in post-treatment acute myeloid leukaemia~\cite{Ding_2012}, glioma~\cite{Johnson_2013}, and esophageal adenocarcinoma~\cite{Murugaesu_2015}. At the same time, there is increasing appreciation for the importance of T cell neoantigens arising from somatic mutations in anti-tumor immune responses~\cite{Schumacher_2015}. Patients whose tumors harbor more neoantigens may have better prognosis~\cite{Brown_2014} and are more likely to respond to checkpoint blockade immunotherapy~\cite{Van_Allen_2015,Rizvi_2015}. It has been suggested that a relatively low neoantigen burden in ovarian cancer may explain the limited effectiveness of certain immunotherapies in this disease~\cite{Martin_2016}, raising the question if chemotherapy might be used specifically to increase neoantigen burden. However, the effect of standard chemotherapy regimes on tumor mutations and resulting neoantigens in both the neoadjuvant and adjuvant settings is unknown.

Investigators associated with the Australian Ovarian Cancer Study (AOCS) performed whole genome and RNA sequencing on 114 cancer samples from 92 patients with high grade serous ovarian carcinoma~\cite{Patch_2015}. Seventy-nine samples were taken before chemotherapy, 30 after adjuvant chemotherapy, and 5 after neoadjuvant therapy. Treatment regimes varied but always included platinum-based chemotherapy. In their primary analysis, Patch et al. reported that post-treatment samples harbored more somatic mutations than pre-treatment samples and exhibited evidence of chemotherapy-associated mutations. Here we extend these results by quantifying the mutations and predicted neoantigens attributable to chemotherapy-associated mutational signatures.

Our use of mutational signatures is necessary because it isn't possible to distinguish chemotherapy-induced mutations from temporal effects when comparing primary and relapse samples by mutation count alone. A mutational signature ascribes a probability to each of the 96 possible single-nucleotide variants, where a variant is defined by its reference base pair, alternate base pair, and base pairs immediately adjacent to the mutation. Signatures have been recently associated with exposure to particular mutagens, age related DNA changes, and disruption of DNA damage repair pathways due to somatic mutations or germline risk variants in melanoma, breast, lung and other cancers~\cite{Alexandrov2013}, and therefore provide a means of identifying the contribution that chemotherapy may make to the mutations seen in post-treatment samples. For example, the chemotherapy temozolomide has been shown to induce mutations consisting predominantly of $C \rightarrow T$ (equivalently, $G \rightarrow A$) transitions at CpC and CpT dinucleotides~\cite{Johnson_2013}. To perform deconvolution, the single nucleotide variants (SNVs) observed in a sample are tabulated by trinucleotide context, and a combination of signatures, each corresponding to a mutagenic process, is found that best explains the observed counts. Mutational signatures may be discovered \textit{de novo} from large cancer sequencing projects, but for smaller studies it is preferable to deconvolve into known signatures~\cite{Rosenthal_2016}, as we do here.

% As paclitaxel/carboplatin therapy following surgery is standard of care, all AOCS relapse samples were exposed to chemotherapy. Therefore, the effect of chemotherapy on mutational burden cannot be statistically separated from those of drift and relapse in this cohort. Mutational signature deconvolution provides an alternative by incorporating outside information on the trinucleotide sequence context biases of particular mutagenic processes. For example, mutations arising due to disruption of mismatch repair have a high number of C to T mutations where the 5' and 3' adjacent bases are both G. To perform deconvolution, the SNVs observed in a sample are counted by trinucleotide context, and a combination of signatures, each corresponding to a mutagenic process, is found that best explains the observed counts~\cite{Alexandrov2013}. Mutational signatures may be discovered \textit{de novo} from large cancer sequencing projects, but for smaller studies it is preferable to deconvolve into known signatures~\cite{Rosenthal_2016}, as we do here.

% In this study, we first compared the mutation, neoantigen, and expressed neoantigen burden of chemotherapy-naive primary samples to adjuvant- and neoadjuvant-treated samples and find that the burden increases  

% The COSMIC Signature Resource curates 30 signatures from pan-cancer analyses. While signatures for exposure to the chemotherapies used in ovarian cancer have not been established from human studies, two recent reports provide data on mutations detected in cisplatin-exposed \textit{C. Elegans} organisms~\cite{Meier_2014} and a \textit{G. Gallus} (chicken) cell line exposed to several chemotherapies including cisplatin, chyclophosphamide, and etoposide ~\cite{Szikriszt_2016}. The SNVs identified in these studies enabled us to define putative mutational signatures for cisplatin, cyclophosphamide, and etoposide, which we subsequently applied to the AOCS data set.

% performed signature deconvolution on the AOCS dataset using these four chemotherapy signatures plus the 30 COSMIC signatures, and estimated the neoantigens attributable to each signature.

% The \textit{C. Elegans} study considered cisplatin and other compounds across a range of DNA repair-deficient knockout models. The \textit{G. Gallus} study examined a number of chemotherapies, including cisplatin, cyclophosphamide, and etoposide, in wildtype \textit{G. Gallus} cell lines.

% The COSMIC Signature Resource curates 30 signatures from pan-cancer analyses; a number of these have known associations with mutagenic processes such as disruption of mismatch repair pathways or ultraviolet light exposure~\cite{364242}. While signatures for chemotherapy exposure have not been established from human studies, two recent reports provide data on mutations detected in chemotherapy-exposed \textit{C. Elegans} organisms~\cite{Meier_2014} and a \textit{G. Gallus} (chicken) cell line~\cite{Szikriszt_2016}. The \textit{C. Elegans} study considered cisplatin and other compounds across a range of DNA repair-deficient knockout models. The second study examined a number of chemotherapies, including cisplatin, cyclophosphamide, and etoposide, in wildtype \textit{G. Gallus} cell lines. Based on the mutations identified in these studies, we extracted 10 signatures for cisplatin in various genetic contexts, 1 signature for cyclophosphamide, and 1 signature for etoposide. We then performed signature deconvolution on the AOCS dataset using these the chemotherapy signatures plus the 30 COSMIC signatures.

% Given these results, we wondered if this dataset could be used to assess the effect of chemotherapy on mutation and neoantigen burden.
% analysis might be extended to quantify the number of neo whether chemotherapy leads to more neoantigens.


