\section*{Background}

Many chemotherapies including platinum compounds~\cite{Hannan_1989}, cyclophosphamide~\cite{Anderson_1995}, and etoposide~\cite{NAKANOMYO_1986} exert their effect through DNA damage, and recent studies have found evidence for chemotherapy-induced mutations in post-treatment acute myeloid leukaemia~\cite{Ding_2012}, glioma~\cite{Johnson_2013}, and esophageal adenocarcinoma~\cite{Murugaesu_2015}. Successful development of immune checkpoint-mediated therapy\cite{Chen_2013} has focused attention on the importance of T cell responses to somatic mutations in coding genes that generate neoantigens~\cite{Schumacher_2015}. Studies based on bulk-sequencing of tumor samples followed by computational peptide-class I MHC affinity prediction~\cite{Lundegaard_2007} have suggested that tumors with more mutations and predicted mutant MHC I peptide ligands are more likely to respond to checkpoint blockade immunotherapy~\cite{Van_Allen_2015,Rizvi_2015}. Ovarian cancers fall into an intermediate group of solid tumors in terms of mutational load present in pre-treatment surgical samples\cite{Lawrence_2013}. However, the effect of standard chemotherapy regimes on tumor mutation burden and resulting neoantigen expression in ovarian cancer is poorly understood.


% , and it has been suggested that the relatively low neoantigen burden in ovarian cancer may partly explain the limited effectiveness of immunotherapy in this disease~\cite{Martin_2016}

% However, the effect of standard chemotherapy regimes on tumor mutations and resulting neoantigens in both the neoadjuvant and adjuvant settings is unknown.

Investigators associated with the Australian Ovarian Cancer Study (AOCS) performed whole genome and RNA sequencing on 114 cancer samples from 92 patients with high grade serous ovarian carcinoma~\cite{Patch_2015}. Treatment regimes varied but primary treatment always included platinum-based chemotherapy. In their primary analysis, Patch et al. reported that post-treatment samples harbored more somatic mutations than pre-treatment samples and exhibited evidence of chemotherapy-associated mutations. Here we extend these results by quantifying the mutations and predicted neoantigens attributable to chemotherapy-associated mutational signatures.

% Investigators associated with the Australian Ovarian Cancer Study (AOCS) performed whole genome and RNA sequencing on 114 cancer samples from 92 patients with high grade serous ovarian carcinoma~\cite{Patch_2015}. Seventy-nine samples were taken before chemotherapy, 5 after neoadjuvant therapy (NACT), and 30 after chemotherapy administered in the adjuvant or metastatic setting (AMCT). Treatment regimes varied but always included platinum-based chemotherapy. In their primary analysis, Patch et al. reported that post-treatment samples harbored more somatic mutations than pre-treatment samples and exhibited evidence of chemotherapy-associated mutations. Here we extend these results by quantifying the mutations and predicted neoantigens attributable to chemotherapy-associated mutational signatures.

In this study, we assess changes in mutation and neoantigen burden after NACT and AMCT, perform signature deconvolution using putative signatures for chemotherapy exposure derived from preclinical studies, and quantify the contribution of each signature on neoantigen burden. We find that neoantigen burden increases after AMCT and relapse but not after NACT, and that only a small part of the increase is attributable to chemotherapy.

