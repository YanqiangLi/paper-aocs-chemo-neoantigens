\section*{Methods}
As paclitaxel/carboplatin therapy following surgery is standard of care, all AOCS relapse samples were exposed to chemotherapy. Therefore, the effect of chemotherapy on mutational burden cannot be statistically separated from those of drift and relapse in this cohort. Mutational signature deconvolution provides an alternative. In this method, the SNVs observed in a sample are counted by trinucleotide context, and a combination of signatures, each corresponding to a mutagenic process, is found that best explains the observed counts~\cite{Alexandrov2013}. Mutational signatures may be discovered \textit{de novo} from large cancer sequencing projects, but for smaller studies it is preferable to deconvolve into known signatures~\cite{Rosenthal_2016}, as we do here.

The COSMIC Signature Resource curates 30 signatures from pan-cancer analyses; a number of these have known associations with mutagenic processes such as disruption of mismatch repair pathways or ultraviolet light exposure~\cite{364242}. While signatures for chemotherapy exposure have not been established from human studies, two recent reports provide data on mutations detected in chemotherapy-exposed \textit{C. Elegans} organisms~\cite{Meier_2014} and a \textit{G. Gallus} (chicken) cell line~\cite{Szikriszt_2016}. The \textit{C. Elegans} study considered cisplatin and other compounds across a range of DNA repair-deficient knockout models. The \textit{G. Gallus} study looked at a number of chemotherapies, including cisplatin, cyclophosphamide, and etoposide, in wildtype chicken cell lines. Using the mutations identified in these studies, we extracted signatures for cisplatin, cyclophosphamide, and etoposide, and included 
