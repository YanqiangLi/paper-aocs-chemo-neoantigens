High neoantigen burden is associated with response to checkpoint blockade immunotherapies. Using computational MHC I binding prediction to identify potential neoantigens, we compared the predicted neoantigen burden of primary Ovarian cancer tumors to matched relapse samples across 12 donors from the Australian Ovarian Cancer Study, 3 donors from TCGA, and 1 donor from our institution. All patients received combination paclitaxel/carboplatin in addition to other therapies. We find a trend toward increased burden in the relapse samples over the matched primary tumors. The mutations private to relapse samples are present at similar allelic fractions as the shared mutations, suggesting they are not especially subclonal. Mutational signature deconvolution did not consistently attribute the new mutations to a known signature not already operative in the primary samples. In particular, the mutations were not attributed to a signature associated with cisplatin exposure in \textit{C. Elegans}.

% These results suggest that the increase in neoantigen burden is increased in recurrent Ovarian cancer over primary samples but that this increase is due to a mix of intrinsic processes as opposed to direct mutagenic impact of the platinum chemotherapy.