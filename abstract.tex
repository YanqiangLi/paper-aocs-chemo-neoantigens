High neoantigen burden is associated with response to checkpoint blockade immunotherapies. We investigated computationally-predicted neoantigen burden in a cohort of 12 donors from the Australian Ovarian Cancer Study with whole genome DNA sequencing of chemotherapy-naive primary tumors and post-paclitaxel/carboplatin treatment relapse samples. We find that the number of potential neoantigens is substantially increased in the post-chemo recurrence samples. The new mutations are found at allelic fractions similar to the mutations originally detected in the primary samples, indicating they are not especially subclonal. Mutational signature deconvolution did not consistently attribute the new mutations to a known signature not already operative in the primary samples. In particular, we did not detect enrichment for a signature associated with cisplatin exposure in \textit{C. Elegans}.

% These results suggest that the increase in neoantigen burden is increased in recurrent Ovarian cancer over primary samples but that this increase is due to a mix of intrinsic processes as opposed to direct mutagenic impact of the platinum chemotherapy.