\subsection*{Background}
Patients with highly mutated tumors, such as melanoma or smoking-related lung cancer, have higher rates of response to immune checkpoint blockade therapy, perhaps due to increased neoantigen expression. Many chemotherapies including platinum compounds are known to be mutagenic, but the impact of standard treatment protocols on mutational burden and resulting neoantigen expression in most human cancers is unknown.

%Patients with highly mutated tumors have better prognosis in a variety of clinical contexts. Part of this effect may be due to an increased number of T-cell neoantigens, resulting in more immunologically foreign tumors. Many chemotherapies including platinum compounds are known to be mutagenic, but the impact of standard treatment protocols on detectable mutational burden and resulting neoantigens in most human cancers is unknown.

\subsection*{Methods}
% Based on chemotherapy-induced mutations from a study of chemotherapy-induced mutations in a \textit{G. Gallus} cell line, a study of c 

Using computational peptide-MHC binding prediction, mutational signature deconvolution, and signatures extracted from two studies of chemotherapy-exposed animal cells, we sought to quantify the effect of chemotherapy treatment on mutation and resulting neoantigen expression in a cohort of 92 patients with high grade serous ovarian carcinoma from the Australian Ovarian Cancer Study.

% Using 115 samples from the Australian Ovarian Cancer Study, we sought to quantify the effect of chemotherapy treatment on mutation and resulting neoantigen burden in this disease.

\subsection*{Results}
Relapse samples taken after adjuvant chemotherapy harbored a median of 90\% more expressed neoantigens than untreated primary samples, whereas neoadjuvant-treated primary samples showed no increase in noeantigen burden over untreated samples. The contribution from chemotherapy-associated signatures was small, accounting for at most 16\% of the expressed neoantigen burden in any sample. The mutagenic processes responsible for most mutations at relapse were similar to those operative in primary tumors, with COSMIC \textit{Signature (3)}, associated with BRCA disruption, \textit{Signature (1)}, associated with age-related spontaneous deamination of 5-methylcytosine, and \textit{Signature (8)}, of unknown etiology, accounting for most of the mutations and neoantigens both before and after chemotherapy.

\subsection*{Conclusion}
Relapsed high grade serous ovarian cancer tumors harbor nearly double the predicted expressed neoantigen burden of primary samples, but adjuvant cisplatin and cyclophophamide chemotherapy treatments account for only a small part of this effect.


% Patients with highly mutated tumors have better prognosis in a variety of clinical contexts. Part of this effect may be due to an increased number of T-cell neoantigens, resulting in more immunologically foreign tumors. Many chemotherapies including platinum compounds are known to be mutagenic, but the impact of standard treatment protocols on detectable mutational burden in most human cancers is unknown. Using 115 samples from the Australian Ovarian Cancer Study, we quantify the effect of chemotherapy treatment on mutation and resulting neoantigen burden in this disease. We find that, while chemotherapy-treated samples have more detectable mutations and neoantigens than primary samples (an estimated increase of 57\% and 65\%, respectively), only a small portion of these (7\% and 5\%) match chemotherapy-exposure signatures derived from studies of chemotherapy-exposed \textit{C. Elegans} organisms and \textit{G. Gallus} cell lines. In both treated and untreated samples, the majority of mutations and neoantigens are instead attributed to signatures associated with BRCA pathway disruption (COSMIC Mutational Signature 3), a process of unknown etiology (Signature 8), and spontaneous deamination of 5-methylcytosine (Signature 1).


% We find that, while chemotherapy-treated samples have an estimated 57\% more detectable mutations and 65\% more potential neoantigens than primary samples, only a median of 7\% (range 0–22) of mutations and 5\% (range 0–15) of neoantigens in the treated samples match signatures derived from studies of chemotherapy-exposed \textit{C. Elegans} organisms and \textit{G. Gallus} cell lines.


% These results suggest that standard chemotherapy regimes are not a dominant source of neoantigens in this disease.

% These results demonstrate an approach for quantifying chemotherapy-associated mutations and neoantigens and suggest that standard chemotherapy regimes are not a dominant source of neoantigens in ovarian cancer.