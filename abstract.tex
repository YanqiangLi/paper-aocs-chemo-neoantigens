There is increasing evidence that high somatic mutation burden, and the neoantigens that result, are favorable prognostic markers for patients in a variety of clinical contexts. Several highly-used chemotherapies are known to be mutagenic, but their contribution to detectable somatic mutation burden in treated human cancers is unknown. Based on a reanalysis of 115 whole genome sequenced samples from the Australian Ovarian Cancer Study, here we show that, while relapsed samples have 65\% more potential neoantigens than primary samples, less than 5\% of the mutations and neoantigens are typically attributable to experimentally-derived mutational signatures of chemotherapy exposure. In an extremely heavily treated patient, chemotherapy-associated SNVs accounted for 22\% of mutations and 15\% of neoantigens. These results suggest that neoantigen induction may not be a rationale for combining chemotherapy and immunotherapies.

% High neoantigen burden is associated with response to checkpoint blockade immunotherapies. Using computational MHC I binding prediction to identify potential neoantigens, we compared the predicted neoantigen burden of primary Ovarian cancer tumors to donor-matched relapse samples across 16 patients from the Australian Ovarian Cancer Study, TCGA, and our institution. All patients received adjuvant paclitaxel/carboplatin chemotherapy in addition to other therapies. We find a trend toward increased burden in the relapse samples over the matched primary tumors, as expected given the increase in overall somatic mutation burden previously reported for these samples. The mutations private to relapse samples are present at similar allelic fractions as the shared mutations, suggesting they are not especially subclonal. Mutational signature deconvolution did not consistently attribute the new mutations to a known signature not already operative in the primary samples. In particular, the mutations were not attributed to a signature associated with cisplatin exposure in \textit{C. Elegans}. This suggests the increased burden at relapse in these Ovarian cancer samples is not predominantly due to direct mutagenic impact of the adjuvant chemotherapy, in contrast to a recent report for neo-adjuvant treated Esophageal cancer.

% These results suggest that the increase in neoantigen burden is increased in recurrent Ovarian cancer over primary samples but that this increase is due to a mix of intrinsic processes as opposed to direct mutagenic impact of the platinum chemotherapy.