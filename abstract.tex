\subsection*{Background}
Patients with highly mutated tumors, such as melanoma or smoking-related lung cancer, have higher rates of response to immune checkpoint inhibitor therapy, perhaps due to increased neoantigen expression. Many chemotherapies including platinum compounds are known to be mutagenic, but the impact of standard treatment protocols on mutational burden and resulting neoantigen expression in most human cancers is unknown.

%Patients with highly mutated tumors have better prognosis in a variety of clinical contexts. Part of this effect may be due to an increased number of T-cell neoantigens, resulting in more immunologically foreign tumors. Many chemotherapies including platinum compounds are known to be mutagenic, but the impact of standard treatment protocols on detectable mutational burden and resulting neoantigens in most human cancers is unknown.

\subsection*{Methods}
Using 114 high grade serous ovarian cancer samples from the Australian Ovarian Cancer Study, we sought to quantify the effect of chemotherapy treatment on mutation and resulting neoantigen burden.

% Using 115 samples from the Australian Ovarian Cancer Study, we quantify the effect of chemotherapy treatment on mutation and resulting neoantigen burden in this disease.

\subsection*{Results}
While chemotherapy-treated samples have more detectable mutations and neoantigens than primary samples (an estimated increase of 57\% and 65\%, respectively), only a small portion of these (7\% and 5\%) match chemotherapy-exposure signatures derived from studies of chemotherapy-exposed \textit{C. Elegans} organisms and \textit{G. Gallus} cell lines. In both treated and untreated samples, the majority of mutations and neoantigens are instead attributed to signatures associated with BRCA pathway disruption (COSMIC Mutational Signature 3), a process of unknown etiology (Signature 8), and spontaneous deamination of 5-methylcytosine (Signature 1).

\subsection*{Conclusion}
TODO discuss subclonal Direct mutagenesis from standard chemotherapy regimes typically contributes a minority of the neoantigen burden in adjuvant-treated relapsed or neoadjuvant-treated primary ovarian cancer.


% Patients with highly mutated tumors have better prognosis in a variety of clinical contexts. Part of this effect may be due to an increased number of T-cell neoantigens, resulting in more immunologically foreign tumors. Many chemotherapies including platinum compounds are known to be mutagenic, but the impact of standard treatment protocols on detectable mutational burden in most human cancers is unknown. Using 115 samples from the Australian Ovarian Cancer Study, we quantify the effect of chemotherapy treatment on mutation and resulting neoantigen burden in this disease. We find that, while chemotherapy-treated samples have more detectable mutations and neoantigens than primary samples (an estimated increase of 57\% and 65\%, respectively), only a small portion of these (7\% and 5\%) match chemotherapy-exposure signatures derived from studies of chemotherapy-exposed \textit{C. Elegans} organisms and \textit{G. Gallus} cell lines. In both treated and untreated samples, the majority of mutations and neoantigens are instead attributed to signatures associated with BRCA pathway disruption (COSMIC Mutational Signature 3), a process of unknown etiology (Signature 8), and spontaneous deamination of 5-methylcytosine (Signature 1).


% We find that, while chemotherapy-treated samples have an estimated 57\% more detectable mutations and 65\% more potential neoantigens than primary samples, only a median of 7\% (range 0–22) of mutations and 5\% (range 0–15) of neoantigens in the treated samples match signatures derived from studies of chemotherapy-exposed \textit{C. Elegans} organisms and \textit{G. Gallus} cell lines.


% These results suggest that standard chemotherapy regimes are not a dominant source of neoantigens in this disease.

% These results demonstrate an approach for quantifying chemotherapy-associated mutations and neoantigens and suggest that standard chemotherapy regimes are not a dominant source of neoantigens in ovarian cancer.