High neoantigen burden and CD8+ T cell immune infiltration are associated with favorable response to checkpoint blockade immunotherapies. We investigated computationally-predicted neoantigen burden and immune infiltration in a cohort of 12 donors from the Australian Ovarian Cancer Study with RNA and whole genome DNA sequencing of chemotherapy-naive primary tumors and combination paclitaxel/carboplatin treated relapse samples. We find that predicted neoantigen burden is substantially increased in the post-chemo recurrence samples, and is not confined to subclonal mutations. Mutational signature deconvolution did not consistently attribute these mutations to a known signature across samples; in particular, no enrichment for a signature associated with cisplatin exposure in C. Elegans was detected. RNA-seq based immune cell deconvolution suggested that the recurrent ascites samples harbor an immunosuppressive environment composed predominantly of monocytes, with fewer CD8+ T cells than the primary solid tumors, an effect which most likely reflects differences in the tissue type but could in principle also be due to treatment or relapse disease biology. These results suggest that neoantigen burden is increased in recurrent Ovarian cancer over primary samples, that this increase is due to a mix of intrinsic processes as opposed to direct mutagenic impact of the platinum chemotherapy, and that different approaches may be needed to elicit an immune response against ascites and solid tumor cell populations owing to the different immune milieu of these tissues.