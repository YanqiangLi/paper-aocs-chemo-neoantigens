\section*{Discussion}

In this investigation of chemotherapy-associated mutations and resulting neoantigens in ovarian cancer, we find that the number of neoantigens after surgery, chemotherapy, and relapse is as much as double that of the primary tumor. However, the number of mutations matching signatures observed in \textit{G. gallus} and \textit{C. Elegans} exposured to cisplatin or cyclophosphamide usually account for less than 5\% of the overall mutation and neoantigen burden after treatment. In the most heavily treated patient, chemotherapy-associated SNVs accounted for 22\% of mutations and 15\% of neoantigens.

% STATEMENT_MEDIAN_NEOANTIGENS
We predicted a median of 64 (50--75) expressed MHC I neoantigens across all samples in the cohort, significantly more than the 6 expressed neoantigens reported in another recent study of this cancer type\cite{Martin_2016}. However, Martin et al did not consider indels, MNVs, or multiple neoantigens that can result from the same missense mutation, used a 100nm instead of 500nm MHC I binding threshold, used lower quality sequencing (50bp reads), and only explicitly considered HLA-A alleles. We suggest that MHC I binding counts such as those presented here should be seen as a relative measure of tumor foreignness. The absolute counts are highly sensitive to parameter choices in the analysis and great care must be taken in comparing across studies.

A surprising finding was that in samples of post-chemotherapy ascitic fluid, the number of neoantigens with evidence for RNA expression tracks the increase in DNA neoantigens, but decreases in solid tumors, despite the overall increase in neoantigens in the DNA. This raises the possibility of immunoediting\cite{Dunn_2002} at the level of transcription occurring post-chemotherapy in solid tumors. This idea is consistent with studies showing that paclitaxel/carboplatin and other chemotherapies may enhance T cell infiltration and cytotixicity in solid tumors\cite{Demaria2001,Wu_2009,Pfannenstiel_2010,Hodge_2013}, whereas ascites remain immune-suppressed\cite{Giuntoli2009,Simpson-Abelson2013,Singel2016}. However, this result is based on only four post-treatment solid tumor samples with RNA sequencing, and the analysis method is sensitive to changes in the expression of unrelated genes. Larger cohorts with paired pre- and post-chemotherapy RNA sequencing of solid tumors are needed to determine if transcriptional downregulation of neoantigens after treatment is a recurring phenomenon.

This study makes several technical contributions, which we hope may be a foundation for future work. We proposed a hierarchical Bayesian regression model to combine paired count data while controlling for confounding variables. We developed a method for extracting mutational signatures suitable for use with deconstructSigs from treated/control mutagenicity studies, and release the nine extracted signatures with this publication. Our neoantigen identification pipeline is also made available, which includes support for indels and the identification of neoantigens with RNA support. 

There are several important limitations to this study. The Bayesian model controlled for the most obvious confounding variables (tumor sample type, purity, and the number of samples from the same donor) but there are potentially other sources of bias to mutation calling, such as variations in read coverage over the genome and the clonal structure of the tumor. The neoantigens resulting from frameshifts, which consider up until a stop codon, may be over-estimated, as it is not clear how many of these would in fact make proteins  instead of nonsense-mediated decay. As previously mentioned, the hard threshold on read count from RNA-seq to consider a neoantigen expressed is sensitive to changes in the expression of unrelated genes. The signature extraction used only SNVs; MNVs and indels were considered separately, and large structural rearrangements were not considered. Addressing these issue would require an extended formulation of mutational signature deconvolution, which is beyond our scope but probably a useful direction. Finally, a substantial fraction of mutations are unresolved by signature deconvolution, and this amount increases with treatment. Some of these unresolved mutations may be a result of chemotherapy-induced mutations that do not fit the animal-derived mutagenic signatures. 

% Large genomic rearrangements were also not considered, and are known to result in neoantigens \cite{Worley2001}.

TODO: some sort of comparison of our neoantigen counts against another study in e.g. TCGA, and perhaps a calculation for how much more likely someone is to benefit from chekcpoint blockade based on these mutations


